\documentclass{article}
\usepackage[utf8]{inputenc}

\title{Science and Life}
\author{Dominic Jones}
\date{August 2016}

\begin{document}

\maketitle

\begin{abstract}
Three fields of scientific investigation are considered which manifestly suffers from an absence of philosophical support when considering their aims. They are: the pursuit of artificial intelligence, evolution and embryonic stem cell research. All three make at least implicit, fallacious assumptions about the actual and speculated achievements the their scientific domain, the extent of applicability of their scientific research, and about the service their science yields to humanity, respectively. All three also have significant coverage in the public sphere, influencing the common understanding of who man is and what is his value.
\end{abstract}

\section*{Introduction}
Natural science seeks to progress the knowledge about the world around us and facilitate us in the domination of it. The three topics considered here fail to acknowledge: that man can only create things which whose causality is deterministic, that life is materially irreducible and its role is active for the form of being and not simply its activity, that the question of life is a central question of modern science and technology. Life, then, is the common theme among all three, but each considering it under a different aspect: as deterministically intelligent, as materially caused, and as non-existent. These three aspects are fallacious aspects of life, but require a Realist philosophy to examine them and to restate the aspects appropriately. This task is not \emph{academic}, science influences public thinking, legislation, attitudes, and civil life in general. To be mistaken on these topics is to hold an utterly false anthropology.

\section*{Artificial Intelligence}
The ``Turing Test'' is a pragmatic test proposed by Alan Turing whereby if a person who is in written communication with some other being cannot tell whether or not the other being is a computer program or a person then the other being possesses intelligence. This test side steps the question of whether or not the being is one which possesses intelligence. Rather it is simply a behavioural test.

Two intense periods of research sought to construct programs which could behave intelligently: the first period was at MIT in the 1980s and the second period is the present day work pursued by Google. On the face of this work there has been much success: early successes were chess programs that can beat the best players. These programs were encoded with the rules of the chess game and with the many known strategies which players adopted. This, combined with sufficient computing power lead to its success. However, more recently there has been a paradigm shift in that the latest programs do not have the rules of the target game encoded in them. The only human intervention in preparing the program is to provide the program the objective to be maximised (i.e. the score) and the inputs that control the game (i.e. the movement of the pieces, as in the case of chess). With the program set up accordingly, it then plays real players, or other programs and once there is a minimum of 'learning' acquired it then plays against itself many millions of times.

The results are impressive. After sufficient learning, these programs can outperform the very best human competition and even present novel strategies. And yet with all this, never once were the rules of the game encoded.

This technology has much appeal but it is pursued by people who conceive of it as intelligence, akin to the intelligence of human intelligence and so applicable in the same manner. To non-specialists, the glamour disguises the fundamentals. It is simply a computer program and computer programs are necessarily deterministic. More technically, it is a computer program which maximises a predefined output by manipulating the inputs.

The philosophical blur is the comparison of the success of these programs to the achievements of man for the same task. Man needs to learn the rules, it does not. Man can perform well in the task, but the program can outperform man. Man is naturally limited to few proficiencies, the programs can learn any number. But assumed by the pursuers of this work is that intelligence is algorithmically representable. This view fails to account for the ingenuity of man which is manifest in his pursuit of knowledge, of his capacity to speculate on the probable success of pursuing a line of enquiry prior to actually committing to the pursuit. Though above all, in this context, it fails to account for the capacity of man's intelligence to grasp rules in a single attempt and set and carry out objectives autonomously.

If, at the heart of the thinking lies the assumption that intelligence is algorithmic, culpability becomes one step removed if such a program is adopted for tasks which grave responsibility accompanies its execution. Indeed, the very notion of culpability becomes nominalistic. Free Will and its accompanying responsibility are replaced by predictably deterministic and auditable steps. But no program can `know', or `be taught' qualitative rules. Programs are numeric operations and everything it does must be somehow reduced to numeric value.

\section*{Evolution}
The theory of evolution, rather like but to lesser degree the Big Bang theory, is a theory often strongly associated with a philosophical view. Prior to the adoption of the Big Bang the 'steady state' cosmos was promoted. This was in light of the consequences of Newton's second law: that the inertia of an object is unchanged unless acted upon, i.e. motion can exist without continued external force. In short, these ideas could be seen as denouncing a Creator God - the universe had no beginning (or more incorrectly, had no creation), or at least promoting a mechanistic world view - God set the world in motion and left it to its own devices.

Evolution would appear to fit well into this cosmic view: the universe may or may not have been created, but even if it was, it was left to its own devices and from it, by chance, came the natural world.

Among the various issues with the theory, man presents a significant counter-argument to the `by chance' coming to be. He appears to have something that is irreducible to matter, something that cannot be measured empirically, namely his intelligence and his free will.

Artificial Intelligence, to some, provides an adequate answer. Evolution, rather like Artificial Intelligence, assumes again that the efficient causality that fully accounts for the final causes obtained resides exclusively in the deterministic natural order. The mystery of man becomes no mystery at all, and life becomes a term to denote complexity in degree of things in the natural world rather than a difference in kind from `pure matter'.

Life, however, is the protagonist in the Evolution theory. If the notion of life is fallacious then the arguments based around Evolution will be badly misguided. What definition is to be given to life? Admittedly, this is difficult to define as there are `hard cases' such as viruses, but a definition is as much a task of biology as of philosophy. Certain ideas of Evolution imply that there is nothing original in the formation of living beings. Chance mutation is the cause of living beings. This is a behavioural definition. By chance something comes to exist that behaves with characteristics that are common to all living beings. This `explanation' has similarities with Artificial Intelligence, namely Reductionism.

Chance, however, has no predictive power and little or no explanatory power. Chance is not an efficient cause. Chance, in the scale assumed by Evolutionary mechanics, for the existence and variety of species, is not science at all but cras denial of activity that lies outside the realm of natural science, namely the active agent of life.

The question of life is the pressing question implicit in the controversy of Evolutionary theories. If life and complex material phenomena are synonymous, then presumably with sufficiently advanced technologies, it can be synthesised and manipulated arbitrarily.

\section*{Embryonic stem cell research}
The materiality of life and the deterministic development of intelligence provide ample argument for the manipulation of human embryos. As this is already a reality, the necessity of providing a convincing universal metaphysics is all the more urgent. A metaphysics that recognises the intelligibility of the natural world by man, but equally recognises that the intelligibility is partial, contextual and open-ended. Also this philosophy must recognise that man participates in the activity of life. He is not and never can be an external observer. What he perceives about life, he perceives principally by reflection rather than by investigation.

\section*{Summary}
Science must leave room for non-deterministic causality. Whilst progress has been seen in mimicking certain aspects of human behaviour, its confines and limits should be noted: it is not intelligence though it may behave like intelligent beings. Second, science must not force an utterly implausible explanation for that which it has no explanation. Science influences public opinion and decision making and so what is presented by science should be that and no more.



\end{document}
