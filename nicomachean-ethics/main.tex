\documentclass[a4paper,10pt]{article}

\usepackage[bottom=1.0in,top=0.5in,left=1in,right=1in]{geometry}
\usepackage[english]{babel}
\usepackage{graphicx}
\usepackage{float}
\usepackage{subcaption}

\usepackage[utf8]{inputenc}
\usepackage{epigraph}

\setlength\epigraphwidth{0.8\textwidth}
\setlength\epigraphrule{0pt}
\renewcommand*{\textflush}{flushright}

\title{What is virtuous friendship?}
\author{Aristotle's Nicomachean Ethics, books XIII \& XI}
\date{Dominic Jones, June 2018}

\begin{document}
\maketitle



\epigraph{\textit{Men cannot know one another ``till they have eaten the requisite quantity of salt together''}}{}
\vspace{-8mm}
\epigraph{\textit{[Friendship is] ``Mutual affection mutually known''}}{}
\vspace{-8mm}
\epigraph{\textit{It is man's duty to flee from wickedness and strive to be good because then he may be friends with himself and may come to be a friend to another}}{}
\vspace{-8mm}
\epigraph{\textit{Good men, being stable in themselves, are also stable as regards to others: for it is the part of the good neither to do wrong themselves nor to allow their friends in doing so}}{}


Before considering what \emph{virtuous} friendship is, in light of Aristotle's work, the forming of the bond of friendship will be explored, as it lays the foundation to examine how it is to be pursued: as a virtue according to the \emph{via media} or in plenitude, how to recognise deficiencies in friendship, and how to know what the personal demands are in maintaining friendship.

The first matter, however, is to justify the use of describing friendship as \emph{virtuous} in the first place. If it is a fundamental natural inclination then it might seem that no associated \emph{habitus operativus bonus} need be considered since mere association would naturally lead to friendship.

Not surprisingly, this is not so in practice. Aristotle considers various associations which are not friendship, but may form the precursor to one, such as kindly feelings and affection. These are not the hallmarks of friendship because they may occur without knowing the person with any significant depth, indicating a transitory quality lacking any rational (i.e. wilful) grounding. To engage the higher faculties, those peculiar to man, sufficient time of companionship must be spent; time in dialogue, in carrying out a common pursuit, granting opportunities to manifest affection in timely and continual ways. All of these may not call upon the effort of engaging the habit of wanting the good; they may indeed flow with little or no effort. Nevertheless, through these various activities what is described as friendship comes to be acknowledged.

Time is needed to come to know the other, not in some idealised or romanticised way, but to know the standards of the other, the interests, the temperament, the ideals. Getting to know the other in this manner will inevitably take time, and it may be found that there is a lack of common interests, few common ideals over the course of time. By appearances, there may not be much going for an acquaintanceship with little shared vision, but the effort of bringing into this relationship some form of shared vision, and so some form of future, opens the door to friendship.

All this activity presupposes a certain degree of goodness in the persons, as `one cannot give what one does not have'. The goodness considered by Aristotle has a certain specific characteristic: that of stability. In man seeking to do the good habitually, the first indications of virtue are pointed at, namely that \emph{virtuous} friendship begins with virtuous persons.

As to what ultimately comes first, the virtuous friendship leading to virtuous persons, or vice-versa, some consideration is given by Aristotle. The bond between parents and children is a form of friendship, specifically that of benevolence on the part of the parents. Whilst a child cannot reciprocate \emph{in kind}, there at least can be \emph{mutual affection mutually known}. This indicates the pervasive nature of friendship and man's radical dependence upon it to mature his highest faculties, and that man by his own devices may never elicit friendship without first being overwhelmed by it.

Whilst Aristotle marks out stability and mutual affection as prominent qualities, further quality is worth highlighting. The minimal sense of the good, that of not doing wrong, in the context of friendship is applied with firmness, in \emph{not allow[ing] their friends to do so}. To \emph{not allow} is imperative, and this is essentially different from mere wishing or wanting the good of another. \emph{Allowing} or \emph{not allowing} implies a certain strength of relationship that can tolerate and bear such commands. It would seem that to Aristotle, that there is a realism to friendship that must be acknowledged and that it will inevitably manifest at times in these ways. Nevertheless, it is in the context of the good that they arise, a good that does not seek to wound, but ultimately to unify.


\epigraph{\textit{The most common sort of quarrels between friends is their not being friends on the same grounds as they suppose themselves to be}}{}
\vspace{-8mm}
\epigraph{\textit{Where there is chance of amendment we are bound to aid in repairing the moral character of our friends}}{}
\vspace{-8mm}
\epigraph{\textit{The friendship of the good is alone superior to calumny; it not being easy for men to believe a third person respecting one whom they have long tried and proved}}{}

Aristotle considers different kinds of friendship, friendships that have as their goal different aspirations. These he describes as friendships of utility, pleasure and seeking the other's good, or benevolence. In one consideration he contrasts the friendship between two boys and its development through into adulthood. Here, he considers the case when one of the friends matures and so too his purpose in friendship, whereas the other does not. A friendship that started out as one of utility and pleasure, a mutual goal, becomes frustrated over the years because what was the mutual goal is no longer the reality of the bond. One seeks to raise the friendship to its more noble ends, whereas the other is content with its incentives from adolescence. This presents a juncture in such a friendship, for both to strive towards a more worthy common end, or to part company.

With a friend who at some point decides on a path that the other sees is not a real good, it is not of friendship to be indifferent about the matter. It perhaps is only in the context of friendship that there is an opportunity to help the other see things differently, as it is only in the context of friendship that the will of another can be won over. Moreover, it is unlikely that both see `eye to eye' on the same things. That, in fact, they may differ greatly on matters that are most fundamental to each other is not uncommon but also is not a hindrance to the forming of friendship. Before each other they have opportunity to express themselves, their convictions, and in doing so, see for themselves, by the very act of expressing themselves, what it is that they do believe.

That one thinks differently on a matter of importance that at first was not realised may well become a moment of tension between friends, with one or the other thinking that this is not the kind of person with whom there is compatibility. But dissonance is an opportunity for dialogue and willing a better good. In one being presented by another a better good, which may also mean a more demanding good, an opportunity to exercise meaningfully ones power to freely choose is elicited. In the course of dissonance, dialogue and change, a living friendship is found, where each brings the other to yet higher planes.

It is under such circumstances that friendship has opportunity to become \emph{virtuous}. For one this may predominantly entail the effort to examine himself, why is he content with friendship as a means rather than an end in itself, why has there not been the expected maturing which is the frustration of the other. Such reflective questions are the beginning of virtue directed towards friendship, but for their fulfilment a desire to rectify and make concrete steps in doing so would really constitute virtue lived. For the other, patience may well be at the heart of his effort, along with learning to look at matters from the perspective of his friend with a spirit of ingenuity and tact, trying to find definite ways to encourage a renewed purpose of friendship.

This example of a dissonance in friendship does not, however, exhaust the opportunities to exercise virtue in friendship. In the case above, there was the matter of differing reasons for the friendship, but even with a friendship having a common goal, even that of benevolence, the moral quality of the friendship is always matter for mutual growth. A friend may fall into a dissolute lifestyle or grow coarsened with compromises in dignity or become cynical in the face of the trials of life. All these constitute the struggles of people, to which friendship would appear to offer its essential and irreplaceable antidote. Aristotle simply conveys this as \emph{repairing the moral character of our friends}, but it is clear that this encompasses a great deal and is often a delicate and long drawn out challenge of friendship.

Growth in the moral quality among friends requires a degree of trust and confidence such that matters of importance and personal consequence can be spoken about, and this means having to reveal things that one would rather not. Taking the risk of trusting another person in some matter, or more likely, in some confidence, provides the impetus for the reciprocal act of entrusting. In one seeing that their vulnerability in their act of trust is honoured, this spurs the other to do likewise, each then entering into a mutually secured relationship which comes to form a tight bond between word, intention and deed.

A proved friendship is alone superior to calumny, according to Aristotle. It is as though because of friendship, exhaustive explanation is not required to ascertain whether or not something told is true. The ascent of the will to believe leans more greatly on the one communicating than on the content of the communication. Without trust, words forfeit their corresponding symbolum of meaning. A friendship \emph{long tried and proved} must have been guarded. Maintaining discretion about what one has come to know in a context of confidence provides a sure means for understanding and valuing the good name of another person. Anyone with a close friend may know factually about potentially defamatory matters and so will realise that the trust placed in him is consequential. And so, such a friend will also know that even to tolerate listening to any calumny would undermine his own integrity. It would seem then that a broad sense of friendship being superior to calumny is to be understood: not merely superior to calumny among a given friendship, but a stable disposition towards all.

In light of this, a forfeit of trust is the most obvious hindrance to friendship, and indeed can be its ruin. From this aspect virtues concerning the moral benefit of friendship become apparent. That each knows how to guard himself, who avoids gossip, and who strives to master a self control such that he knows when and to whom things should or should not be said.


\epigraph{\textit{Of friends there is a limited number, those with whom it would be possible to keep up intimacy; this being thought to be one of the greatest marks of friendship}}{}
\vspace{-8mm}
\epigraph{\textit{To be a friend to many people, in the way of the perfect friendship is not possible; just as you cannot be in love with many at once; it is, so to speak, a state of excess}}{}
\vspace{-8mm}
\epigraph{\textit{Distance has in itself no effect upon friendship, but only prevents the acting out: yet, if the absence be protracted it is thought to cause a forgetfulness even of the friendship}}{}

Friendship, in its highest sense, is \emph{a state of excess} according to Aristotle, indicating that in fact it is not a virtue of the \emph{via media} but of plenitude. Friendship is not something tempered by some opposing tension in order to give it balance, like that between justice and mercy, but rather that as friendship seeks out its closest companions, it necessarily narrows in breadth, as \emph{it would be impossible to keep up intimacy} otherwise. Indeed, friendship would seem to have its apex in the exclusive other. At this stage, however, what has become of \emph{virtue}? There would seem to be no obvious demand laid upon friends of this degree.

Aristotle considers one source, however. The loss of separation, felt acutely among close friends, may give way to forgetfulness without the continual effort to make present the friendship. How the maintaining of such friendships are to be `kept in good repair' is not eluded to, and, for Aristotle, ultimately may have seemed like an impasse.

Nevertheless, any protracted or continual separation may come to lack its original vitality. The original mutual affection may become tainted with the concern that the other may not be like `the person I knew' anymore, or mistrust that one or the other may no longer believe mutual affection persists.

Despite the difficulties of separation, a strong bond of friendship can bear such loss when the innermost sentiments, the hopes and sorrows, formed the heart of the relationship. That there had been exchange of trust is not readily forgotten or trivialised in the memory of each, and `picking up from where we left off' needs no prelude of recalibration.

In wanting to \emph{keep up intimacy}, over a long period of time, this may become dampened with want of enthusiasm, thinking that all there is to be known about the other is known and there is little left by way of `the mystery of the other'. To this, the response of the virtuous is to look again, as complacency is readily due to one's own indifference, or loss of sense of wonder. That Aristotle placed wonder as the beginning of philosophy, it may not be too much of a stretch to place it at the heart of friendship, a prerequisite to intimacy. And, like Socrates, who said of himself that \emph{scio me nescire}, one may too come to acknowledge that in friendship, despite knowing someone, there remains a gulf of the unknown.


\end{document}

~

Aristotle generally appears to place the greater emphasis on the intellect than the will in the growth and maturation of friendship, that \emph{the acting out} is not an essential once a friendship is formed. This is not so unlike that of Socrates affirming the pursuit of knowledge as the hallmark of virtue. Clearly, for Aristotle there is the definite effort in friendship in its earlier stages but no strong sense in mature friendships of virtue that engages the will.
