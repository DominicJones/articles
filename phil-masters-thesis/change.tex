\chapter{Change}
\label{ch:change}


This chapter aims to show that change is ontologically real and that there must be composition in change in order for it to be intelligible. Interpretations of Hercalitus and Parmenides are explored to assess the extreme positions on change; the former considering permanence as an illusion and the latter considering change as an illusion. The discussion will seek to show the valuable philosophical insights of their arguments, but nevertheless will go on to show that their arguments are ultimately untenable.

\section{Does everything change?}

The assertion that for all material things `everything changes' will be examined under two interpretations in the two following sections. The first interpretation is the claim in the absolute sense, namely that there is no principle of permanence whatsoever. For this interpretation, the critique of Feser will be presented to show that the position is untenable. The second interpretation is a more nuanced position, attributed to Cohen, whereby change is governed by orderly activity, so even though `everything changes', the position introduces some principle of permanence, the permanence of differing kinds of change. It is this orderly activity that provides the principle of unity to things.

\subsection{Nothing persists through change}

\textcite[][13--20]{feser2019aristotle} interprets Heraclitus' position on change in an unqualified sense, as the view that \emph{everything} changes. This is a standard interpretation, even if it is not explicit in one of the more popular claims of Heraclitus: `in the same river, we both step and do not step, we are and we are not' \autocite[][29]{fitt1983ancilla}. Rather, it is the received doxography from Plato:
\begin{quoting}
Heraclitus, you know, says that everything moves on and that nothing is at rest; and, comparing existing things to the flow of a river, he says that you could not step into the same river twice. (Cratylus 402A)
\end{quoting}
In nature there are changes in the material composition, location and quality of inanimate things, and in animate things further changes of conception, growth and corruption can be identified. The straightforward sense of `everything changes' is that indeed, everything is undergoing continuous change in some way or another. Described in this manner, there would appear be be nothing controversial about `everything changes'. But, a point of contention rests on whether or not change can take place in all respects at any given time. To claim that there is complete continuous change in natural things, whereby everything from one moment to the next does not inherently preserve any continued identity or principle of stability, is the position Feser identifies with Heraclitus.
\begin{quoting}
Heraclitus [...] hold[s] [...] that there is no unity to the stages of the objects that common sense supposes exist, but only the multiple stages themselves. [...] There are no abiding objects of any [...] sort, but just various kinds of series of stages that we mistakenly suppose add up to persisting entities. \parencite[][17]{feser2019aristotle}
\end{quoting}

One might suppose that it will be possible to defend the position that everything changes by appealing to the rate at which something changes --- especially if it is something relatively slow with respect to human perception. However, Feser argues that this will not work.
%% Defending the position that `everything changes' by appealing to the rate at which something changes {\textemdash} especially if it is something relatively slow with respect to human perception {\textemdash} would, for Feser, offer no benefit.
That a tree changes slowly, and so can be recognised and identified despite undergoing change, is not a kind of argument that could be appealed to for several reasons. Continuous change means that from one moment to the next there is nothing, in principle, that persists. Therefore, appealing to the timescales of changes in natural things as showing some degree of persistence does not respond to how, in principle, something actually retains any persistence. It is therefore to beg the question. Second, retention of a concept from one moment to the next by the person claiming to recognise persistence in continuous change would require some account of why concepts are exempt, in principle, from continuous change.
If `everything changes' were true, `then there would not be such a thing as a single abiding mind which holds together long enough to [...] even formulate the view.' \parencite[][18]{feser2019aristotle} Furthermore, continuous change has to appeal to certain {universals} in order to formulate the position.\footnote{Universals are a class of mind-independent entities, usually contrasted with individuals (or so-called `particulars'), postulated to ground and explain relations of qualitative identity and resemblance among individuals. \parencite[][]{iep-universals}}
\begin{quoting}
For example, that there is the redness and roundness of a certain ball we experience at one moment, the redness and roundness of the ball we experience at the next moment, and so on, but really no such thing as the ball itself in the sense of a single persisting object that underlies these stages. \parencite[][18]{feser2019aristotle}
\end{quoting}

Everyday experience tells us that things do in fact retain enough of a structure for us to regard them as being stable. If it is granted that at some level there is continuous change, like the continuous motion of atoms making up a living thing, there would appear, sitting at some level between those atomic movements and the changes in growth, etc, that there is indeed some kind of persistence. As to whether or not there is persistence, rather than simply a relatively greater stability, could be seen as a pedantic question. After all, there is nothing that is not capable of corruption in the natural world. An appeal to some principle of permanence is surely an appeal to something obviously non-existent. Furthermore, there appears to be a certain sense of arbitrariness about which level could be taken as the reference level of persistence. Between an acorn and its growth into a mature oak there is no macroscopic appearance of persistence, an acorn does not look anything like an oak (at least until its fruit appears). At the atomic level, the proportion of the acorn that still constitutes the oak is at best negligible. A scientific response to the perceived persistence might appeal to the persistence of its DNA, even though this is something copied (and not necessarily perfectly) rather than preserved. Nevertheless, taking DNA as the reference of permanence could be argued to be just the least arbitrary choice, rather than an identification of permanence.

Despite the empirical evidence weighing against there being some material principle of permanence in the natural world, to go on to conclude that there is, therefore, only continuous change would undermine the the requirement for existence to persist through change.
\begin{quoting}
{[Heraclitus]} is implicitly assuming that there is no single entity underlying and tying together the stages we associate with a thing because he is implicitly assuming that there is only ever potentiality that never congeals into actuality, and thus nothing with the kind of reality that could count as a stable object. But in fact, since all potentiality is grounded in actuality, there could not be change in the first place unless there were some actuality stable enough to ground the potentialities that change presupposes. \parencite[][19]{feser2019aristotle}
\end{quoting}
Consider the analogy of a motion-detection light. Whilst there is no motion the light is off, but nevertheless, \emph{the capacity to change state} is actual (to use the metaphysical term). The light is potentially on, triggered by a potential movement. But an actual movement alone does not trigger the light, even though it is necessary; actual motion detection must be operating continuously. Potentials, then, may become actual, but only with the backing of an already actual state of affairs. Furthermore, whilst the change of state relates to a light turning on and off, there is no such requirement for the persisting cause to be of the same kind. Some electrical circuit powers the motion detection and potentially the light, and this persisting principle is characteristically measured in Volts and Ohms. On the other hand, that which changes is characteristically measured in Lumens.

Whilst the electric circuit ultimately backs the production of light, it nevertheless is a different kind of thing, manifested by its different base measurement units. Relating the analogy to change, even though the perceived change is of a certain kind, namely material, this does not mean that it could not be backed by a persisting principle which is of a different kind, albeit not radically unrelated to the principle that does in fact change. If change is to be understood in this way, then to be ever in the state of becoming something is to ever be in the state of not being anything. Hence something, in principle, must persist, thereby arguing for a compositional account of change.

Furthermore, for change to be intelligible, the range of change could not be unbounded. Potentiality does not mean some principle having the capacity \emph{to become possibly anything}, but rather \emph{to become a range of things}. Returning to the motion-detection light, the operation of the whole unit would be unintelligible if, instead of triggering a light upon motion detection it triggered a jet of water, assuming the principle of actuality remained the same, namely the electrical supply. Whilst that which can be measured in Volts and Amps can be changed into that which is measured in Lumens, it cannot be changed into that which is measured in Kilograms and having the various properties of water. Electricity can be converted into many things, but not \emph{anything}. This position may be denied by appealing to the equivalence of energy and mass, by the relation $E = mc^2$, whereby anything can become anything else (or at least electricity could, in principle, be converted to water). Even so, energy and mass are convertible \emph{in some particular manner}. One Joule of energy is not convertible to one Kilogram of mass, but at least $8.99\,x10^{16}$ Joules are.

For any change to come about, a potential must be actualised, and this must be by something that is already actual, according to the principle of causality. \parencite[][32]{feser2017five} In the extreme Heraclitean view that change can take place under all aspects, however, this points towards notions of continuous creation and annihilation rather than change.
%% \hl{vis: if we do not count mere creation and annihilation as change, H has no change.}
In this scenario the principle of causality has no direct application, as there is nothing that serves as a proximate actual cause to bring about change. Therefore, if change under all aspects is to be posited, some external agent of change would be required in order to satisfy the principle of proportionate causality. This second principle requires that  whatever is found in the effect must be found in the totality of the cause, in some way or other, be it formally, virtually, or eminently.

\begin{quoting}
In scholastic terminology, an effect is contained formally in a cause, when the same nature in the effect is present in the cause: fire causes heat, and the heat is present in the fire. An effect is virtually in a cause when this is not so, as when a pot or statue is caused by an artist. An effect is eminently in a cause when the cause is more perfect than the effect: God eminently contains the perfections of his creation. The distinctions are part of the view that causation is essentially a matter of transferring something, like passing on the baton in a relay race. \parencite[][143]{blackburn1994oxford}
\end{quoting}

Radical change rules out the effect being found in the cause formally, like the growth of a tree whose `oakness' persists through the change. It also rules out the effect being found in the cause virtually, like a tree being chopped down to be used as fuel for a wood burner, since a capacity persists despite the form of `oakness' corrupting. Beyond this, only an external agent serving as the eminent cause could explain the totality of change, along the lines of continuous divine intervention. If this third possibility of eminent causation could be countenanced, the action of divine intervention would also be needed to make the change intelligible to a human intellect, since there is no reason why an exclusively divine action ought to be within human intelligibility. To defend such a line of reasoning, the permanence of the divine intervener would need establishing, thus bringing the argument back to Feser's critiques.


\subsection{Orderly change persists}

%% \hl{Introduce the section: In this section I consider an alternative interpretation suggested by Cohen...}
In this section Cohen's interpretation of Heraclitus' assertion that `everything changes' will be considered.
Doing so is not for the sake of offering a more authentic account of Heraclitus, although it would appear to be, but rather in order to identify some characteristics about change and permanence. Feser argues that there must be, in principle, something that makes things be at any moment in time. Cohen's interpretation does not attempt to deny this, but rather to present considerations on what makes something persist at any moment in time. Cohen presents a more subtle view of Heraclitus, acknowledging his insight that `nature likes to hide.' \autocite[][34]{fitt1983ancilla}

Cohen's reading of Heraclitus is that he does not deny there are persisting objects but rather that an object does indeed undergo continuous change in some respect or other, but not in all respects throughout any change. \parencite[][]{cohen-heracli} Based on his interpretation, a passable but much less extreme position of Heraclitus could be stated as change is continuous for all things in the natural world, but nothing changes in all respects for any given process of change.

Heraclitus' thought on change, as understood by Cohen, concerns the coexistence of opposites in a unity. Based on the earlier remark attributed to Heraclitus that `you could not step into the same river twice' it could reasonably be concluded that he thought that change in quantity, the water that flows down the river, was the principle cause of change, and that static structure was the principle cause of stability through change, like the river banks directing the flow. In fact, Heraclitus seems to consider that underlying activity or change, which may not be immediately obvious and may only be recognised upon reflection, is the bond of unification of an object itself, rather than at least something being held in common under change.
%%
%% Cohen cites the example from Heraclitus of a drink whose ingredients readily separate if the mixture is not stirred: `The ``mixed drink'' (Kyke\^on: mixture of wine, grated cheese and barley-meal) also separates if it is not stirred'. \parencite[][34]{fitt1983ancilla}
%%
Cohen cites the example from Heraclitus of `the ``mixed drink'' (Kyke\^on: mixture of wine, grated cheese and barley-meal) [that] separates if it is not stirred'. \parencite[][34]{fitt1983ancilla}
%%
It is a kind of continuous activity, in this case of mixing, that keeps the drink as such in existence, not merely its ingredients being arranged and set once and for all as though the stability of the drink persisting implies the reaching of a static equilibrium. Persistence of something, then, is itself an act of change, change that moves mere collections of things into some single coherent whole. Furthermore, it is not an incoherent kind of change that provides the unity, but rather an orderly kind of change, such as that of stirring, for example. The insight of Heraclitus is to posit orderly change as the bond of unity, or stability, that stands under any other kinds of change. Whilst the agent of orderly change is left unclear, it is common to the whole of the natural world:
\begin{quoting}
This ordered universe (cosmos), which is the same for all, was not created by any one of the gods or of mankind, but it was ever and is and shall be ever-living Fire, kindled in measure and quenched in measure. \parencite[][27]{fitt1983ancilla}
\end{quoting}

The dynamic principle of persistence through change is further exemplified by Heraclitus in things that have the appearance of being static and unchanging. In his examples on unity in opposites, he considers a stringed instrument whose structure maintains the strings in tension. Here, the very act of maintaining the balance of tension is the action of opposites in a unity. Excessive pull on the strings would collapse the structure of the instrument, whereas the resistance of the structure retains the integrity of the instrument.
\begin{quoting}
They do not understand how that which differs with itself is in agreement: harmony consists of opposing tension, like that of the bow and the lyre. \parencite[][29]{fitt1983ancilla}
\end{quoting}

Under Cohen's interpretation of Heraclitus, matter does not play a crucial role in the persistence of things undergoing change. If the quantitative make up of something were to be utterly replaced, so long as it is through orderly change, there is persistence of the thing itself. Nothing about Cohen's reading of Heraclitus challenges Feser's rebuttal of an extreme interpretation. What it does offer is a consideration about the manner of change, that it is something orderly. The emphasis on order as appose to static structure, for example, exposes change to not simply be something exclusively in the ontological order accessible to rational and non-rational beings, but something that requires the intellect to perceive order so as to recognise change. This kind of view is typical to a Scholastic account of change:
\begin{quoting} \label{quote-torre-change}
The sensible world is the world we see through our senses, while the intelligible world is the world we see through our intelligence. When we `see' being, we see it with our intelligence, not just with our eyes. The eyes do not see being, they only see colour or shapes. But the intelligence sees not only being (and substance and accidents) but also change. Change as such is only intelligible, not sensible. It is a concept the intelligence makes out of (a) the concept of stage before the change, and (b) the concept of stage after the change. The former is ability-to-be or capacity-to-be: potency or potentiality; and the latter is the being-in-act or actuality. Change is the passage from (a) to (b). The being of something changeable is not only what it actually is, but what it can still be. \parencite[][48]{de1981christian}
\end{quoting}
It would be a stretch to imply that notions of Aristotelian hylemorphic composition can be found in Heraclitus' account of change under the interpretation of Cohen, but there would appear to be the beginnings of such a view. What makes a person the same person throughout his life, despite an almost total change in matter? Orderly change might well be the answer favoured by Heraclitus; form would be the answer proposed by Aristotle. They are not the same thing, but form does, among other things, impose order.


\section{Is change an illusion?}

%% \begin{quoting}
%% Nor is Being divisible, since it is all alike. Nor is there anything (here or) there which could prevent it from holding together, nor any lesser thing, but all is full of Being. Therefore it is altogether continuous; for Being is close to Being.\parencite[][45]{fitt1983ancilla}
%% \end{quoting}

%% \hl{Two arguments? only things that can exist can be thought about so coming to exist is unthinkable therefore impossible; something cannot come from nothing so coming into existence is impossible}

A standard interpretation of Parmenides, as expounded by Cohen, is that change is impossible, since only things that already exist can be thought about (first premise), and if something does not already exist it cannot come to exist since something cannot come from nothing (second premise). Change, therefore, must be illusory. Parmenides' argument entails that `it is possible for $x$ to exist if and only if it is possible for $x$ to be thought about, i.e. if and only if $x$ is conceivable.' \parencite[][]{cohen-parm1}.
\begin{quoting}
Come, I will tell you {\textemdash} and you must accept my word when you have heard it {\textemdash} the [two] ways of inquiry which alone are to be thought: the one that \emph{it is}, and it is not possible for \emph{it not to be}, is the way of credibility, for it follows Truth; the other, that \emph{it is not}, and that \emph{it} is bound \emph{not to be}: this I tell you is a path that cannot be explored; for you could neither recognise that which is \emph{not}, nor express it. \parencite[][43]{fitt1983ancilla}
\end{quoting}
\begin{quoting}
For it is the same thing to think and to be. \parencite[][43]{fitt1983ancilla}
\end{quoting}

The `two ways' of Parmenides directly challenges the position of Heraclitus, be it the more extreme interpretation of Feser, or the more subtle of Cohen. Whether it is `that at every moment, every object is changing in every respect', or `that at every moment, every object is changing in some respect or other', change is an illusion. His two ways suggest that what can be known must exist, whereas what is not cannot be known and so cannot be investigated. The position has a resemblance an idea of Plato's, where he states that `speaking is always speaking of something, {...} there is no such thing as speaking of what is not.' (Sophist 237BD--E)

That change is impossible is a conclusion of Parmenides' thought, and so are the conclusions that there is no coming into existence or ceasing to exist, no movement and, arguably, no plurality. Change and movement would require temporal differences, but Parmenides rejects this possibility as this would require some state of affairs going out of existence and some new state of affairs coming into existence. There can be no coming into existence of something as that would imply a time when it did not exist, and to recognise this possibility would be to recognise the possibility that something can be conceived of without existing. As to the conclusion that there can be no plurality, interpretation is not unanimous: Feser interprets the position as one of static monism, whereas Barnes argues that monism is not a feature of Parmenides' thought.\footnote{The denial of change is what makes static monism \emph{static}.} (\citeyear[][163--64]{barnes2002presocratic}) Cohen queries `why can't there be a world of \emph{many} ungenerated, unchanging, indestructible things?' (\citeyear[][]{cohen-parm1}) He, nevertheless, speculatively formulates ways in which Parmenides may have defended monism. If there were two things, $a$ and $b$, in order to differentiate them there must be some property $F$ that $a$ possesses but $b$ does not. However, saying, or `conceiving', that $b$ does not have $F$ is impossible, since one cannot conceive of what does not exist.

Despite the extreme nature of this position, it is one that needs to be responded to. Earlier on, radical change, or pure possible being, was dismissed by Feser since some account of actual being is needed. Cohen's interpretation of Heraclitus offered a principle of persistence, orderly change, but still without elaborating on the origin of the principle of order in `orderly change'.
%% Without this principle being accounted for, interpretations of Heraclitus lack a proper grounding. But,
If a principle of order is posited, it must be more fundamental than that which undergoes the ordering, and, furthermore, must not itself be dependent on orderly change itself. The position of Parmenides, then, is arguably a more fundamental level of reality than that of Heraclitus because the latter is attempting to give an account of an underlying and fundamental principle. Because of this, there is a sense in which it has considerable accordance with a typically Scholastic view of reality, that \emph{to be}, or being, really is the most real thing that we can know, and it is knowable only by abstracting away all association to matter. As \citeauthor[][]{de1981christian} writes:
\begin{quoting}
Types of science can be classified according to the various degrees of elevation above matter. [...] [Matter] designate[s] the world we perceive through the senses, [...] [and] its first characteristic is that it is in continuous change. [...] But science is a stable knowledge, one that does not change. Since matter is the ground of change, in order to reach real scientific knowledge of reality \emph{we have to rise above matter} in order to discover general and stable patterns. [...] [A] second degree of elevation above matter studies beings which although cannot exist without matter can be thought of or conceived without matter, the science called mathematics. [...] [T]he third and highest level of abstraction above matter considers beings which can both exist without matter and be conceived without matter. At this level \emph{we are looking at being as such}. [...] To focus on being requires, therefore, the highest separation from matter. This separation is not a flight from but a much more \emph{pervasive penetration into reality}. It is the level of metaphysics.
(\citeyear[][44--45]{de1981christian})\footnote{Emphasis added.}
\end{quoting}

Parmenides' position involves a tight coupling between the intellect of the knower and the world that he can come to know, a coupling that appears to be exclusively speculative on the part of the intellect with respect to things (or `thing' if a monist reading were assumed). The view of Parmenides seems to be that the intellect can know and should know the things of the world, but there does not appear to be any reality to the creative capacity of the human intellect, such as possessing the intention to fabricate a vase from a block of clay, nor does there appear to be a reality to natural things changing. For Parmenides, the state and extent of reality is bounded and closed irrespective of supposed contributions made by human ingenuity and activity.

In order to offer some corroboration of Parmenides' argument, the limited power of human activity could be considered under two aspects that favour a bounded and closed reality.
The first is the challenge of a thought experiment to conceive of something that bears no association to anything of experience. It is reasonable to say that we cannot.
\begin{quoting}
For, as a matter of fact, painters, even when they study with the greatest skill to represent sirens and satyrs by forms the most strange and extraordinary, \emph{cannot give them natures which are entirely new}, but merely make a certain medley of the members of different animals; or if their imagination is extravagant enough to invent something so novel that nothing similar has ever before been seen, and that then their work represents a thing purely fictitious and absolutely false, it is certain all the same that the colours of which this is composed are necessarily real. \parencite[][7]{descartes-meditations}\footnote{Emphasis added.}
\end{quoting}
Whilst we can think of mermaids and centaurs and regard them as original, it would perhaps be more accurate to describe them and mere imaginary compositions of real things: woman-fish and man-horse. Similarly, even with a vivid imagination of these composites, it is not so obvious what the content of the thought would consist of beyond the mental image. How does the mermaid breathe underwater? Tentative answers could be given, but even with only a superficial enquiry about imaginary compositions, this would be enough to derail a claim that the composition itself is somehow real, or that it involves more than a conjuring of things previously experienced.

The second consideration is the manifest inability to fabricate something with its own intentionality. On the one hand there is the common-sense reality than humans make things intentionally, fabricating them from other things around them. But on the other hand, we cannot make anything properly original, having its own inherent tendencies. Whilst making pots out of clay shows our own intentionality applied to things lacking intentionality, we seem to be unable to impart the intentionality that we possess. We can imitate intentionality in computer programs that `dialogue' with its users and claim that an intentional being has been fabricated. However, to claim intentionality has been imparted, at least in this kind of example, has been refuted by \textcite[][]{Searle1980-SEAMBA}. If something is to be brought into being in a non-trivial sense, a manner that is significantly different from the rearrangement of what already exists, the imparting intentionality appears to be such a standard to distinguish the difference.

These two considerations exemplifying the limitations on the human ability to perform a creative act, be it \emph{conceiving} something new or imparting intentionality in fabricated things to \emph{create} something new, emphasising the same kind of limitation. Both point to powers of composition but deny the possession of creative powers; neither conceiving nor creating seem truly possible. What is possible is composing thoughts or things in some way or another, but neither could be considered the touchstone of reality. What is real in a primary sense are the things of nature. From these two considerations, rejecting change as the movement from something to something that is as yet non-existent does appear to bear some plausibility, insofar as radical novelty can neither be thought about nor fabricated.

To view the position of Parmenides, then, in a way that acknowledges common-sense experience, change may be regarded in two senses. The primary sense being creation and annihilation, which he argues never occurs, and the equivocal sense, which is commonplace, that imagination (or, more generally, thought) and fabrication do occur but they are not change as such, but mere compositions or rearrangements of existing things.
\begin{quoting}
The idea of `arrangement' is how Melissus, follower of Parmenides, describes change: `If a thing changes in any respect, it is rearranged; if it is rearranged, a new arrangement comes into existence. But nothing can come into existence.' \parencite[][]{cohen-parm1}
\end{quoting}
Conception and death do not neatly fit into either of these as neither creation-annihilation nor fabrication-dismantling captures the qualities of something coming to be from pre-existing things, namely parents, but not due to fabrication. Nevertheless, a reasonable Parmenidean-like approach might put emphasis on the idea of the changelessness of species as what is real, and its propagation in time as appearance of change. This idea is, in part, found in Cohen's interpretation:
\begin{quoting}
A more plausible line of argument might go like this: Parmenides thinks that the world is devoid of movement or qualitative change [...]; indeed, there cannot be difference of any \emph{kind}. (\citeyear[][]{cohen-parm1})\footnote{Emphasis added.}
\end{quoting}


The argument of Parmenides that `all that is is being, and changeable things are illusory' is dismissed by Cohen, following Plato, when he points out that not all denials are denials of existence, for `when one says that cows don't fly, one is not referring to flying cows, and saying of them that they don't exist. One is referring to cows, and saying of them that they don't fly.' (\citeyear[][]{cohen-parm1}) Despite this logical error, Parmenides does plausibly insist on a fundamental premise, that nothing comes from nothing. In this premise  are the origins of a number of metaphysical principles, such as the principle of non-contradiction, and the principle of proportionate causality.\footnote{The principle of non-contradiction states that a thing cannot be and not be at the same time in the same respect or relation. The principle of proportionate causality states that the effect cannot be greater than the cause. \parencite[][31]{wuellner2011summary}}
It is because Parmenides grounds his position in at least some well established philosophical axioms that a greater degree of work is needed to refute his argument than that of Heraclitus. Nevertheless, Feser does refute Parmenides, but for different reasons to Plato. Applying the `inverse' of the rebuttal he uses against Heraclitus, he argues that change must be real in order to entertain the very argument that change is an illusion.
\begin{quoting}
For Parmenides to work through the steps of his argument, he has to entertain its initial premise, then to entertain its succeeding premises, and then to entertain its conclusion. He will also thereby have gone from believing that change is real to wondering whether it is in fact real, and finally to being convinced that it is not real after all. But all of that entails the existence of change. (\citeyear[][14--15]{feser2019aristotle})
\end{quoting}
Feser, in offering an Aristotelian response to the problem of change, points out that it is mistaken to understand change as a movement from being to non-being, or vice versa. Change assumes being, as already argued by Torre.
\begin{quoting}
Parmenides held that change entails being arising from non-being, which is impossible. The Aristotelian agrees that it is impossible for being to arise from non-being, but denies that that is what change involves. Among the things having being, we can distinguish actualities and potentialities. [...] Given that a potential is something really in a thing, a change to the thing involves being of one sort (in the traditional jargon, being-in-potency) giving rise to being of another sort (being-in-act), rather than non-being or sheer nothingness giving rise to being. \parencite[][15]{feser2019aristotle}
\end{quoting}


\section{At least some things change}

In examining the positions of Heraclitus and Parmenides, the conclusion that at least some things change has been shown to be unavoidable. At least the person coming to adopt the position $x$, where $x$ may indeed be anyone of the variant arguments on change, or any argument at all, at least he must undergo change in some way, from not holding the position to holding it.

But not everything can be in the state of change. At least something has to be prior to a process of change, possessing some degree of actuality. If reality was purely possible being, nothing would serve as a terminal for change. Despite the logical certainty of this position, exemplifying the difference between a process of change and the beginning and end (the terminals) of change appears to be just outside of our grasp. Change cannot be pointed to or divided up endlessly into partial changes for examination, since `change as such is only intelligible, not sensible'. \parencite[][48]{de1981christian}
What about the `changing of mind'? Since this was a principle route for reaching the conclusion that not everything can change, perhaps this avenue could offer some way of discriminating between change and terminals of change.
Is there a real state of mind that actually exists in the activity of going from `not knowing that at least some things change' to `knowing that some things change'? A possible answer could be `yes, entertaining the possibility but without assenting to its truth.' But this would only push the problem back a stage without actually answering it, since the movement from entertaining to assenting is essentially the same issue.
There does appear to be something paradoxical about the process of change, since in some sense it does not appear to be a `process' at all when a middle ground apparently cannot be admitted. Nevertheless, there must be terminals of change.

As has already been mentioned, the thesis that there must be change as opposed to no change whatsoever can be supported by an appeal to a person changing his mind.
%% That there must be change as opposed to no change whatsoever has the already mentioned appeal to a person changing his mind.
However, there is an asymmetry between the extreme of Heraclitus and the extreme of Parmenides, one which substantially favours Parmenides. The weakness of Heraclitus' argument is that it made no account of what it is that imparts `orderly change', implying that even if there is change, something must govern it. Change, even for Heraclitus, under Cohen's reading, is derived, a product of orderly change. Parmenides, on the other hand, seeks to understand that which is fundamental, and in the process dismisses change as illusory. Whilst it has been argued that it was mistaken for Parmenides to dismiss change, he nevertheless dismisses something that is derived rather than something that is fundamental, unlike Heraclitus (arguably). It is conceivable, therefore, that Parmenides' argument could be true, so long as there was nothing possessing a capacity to change, and this is conceivable insofar as change is a derived reality. It is for this reason that there is more strength to Parmenides' position than to Heraclitus'.

Parmenides insists that `only what exists can be conceived of' as a central claim. Two kinds of examples were given earlier that offered partial corroboration of this claim, considering the limited human capacities of imagination and fabrication. But, could there be a being that does not have such limitations of thought and fabrication such that conceiving of and bringing into existence are univocal terms? Could there be at least one being for whom it is impossible to think about what does not exist? If such a being were to exist, much of Parmenides' onotology would be tenable, since it is plausible that this same being could not be subject to change itself. A sketch of a supporting argument could be: since the limited human capacity can conceive of change, then so could a being with greater conceptual capacity. Furthermore, in this being conceiving of change, change would also come into existence. Since coming into existence is the most radical sense of all possible change, the principle of persistence through such a change must exist independent of all possible kinds of change, including the most radical sense. But to persist through all possible kinds of change just is to persist outside changeable reality. Following on from this, all the concepts, and therefore, realities, must exist outside a changeable reality.

The question of the ontological reality of change strongly suggests the need for some principle that can support change. Feser introduces the solution of principles of potency and act from Aristotle's metaphysics as a solution which captures the essential characteristics of both Heraclitus and Parmenides. For the former: that everything is changing, but with the qualifications of `but not in every way' and `at least in the natural world', and for the latter, that there must be a principle which is not undergoing change, but with the qualification `at least in order to support the reality of change.'

The next chapter will develop Aristotle's theory of act and potency in two stages. First, by starting with natural things, in a similar fashion to Heraclitus, and developing the notion of change being `present in' bearers of change through the categories, or predicaments, of Aristotle. The second stage seeks to draw a relationship between change and bearers of change with potency and act, elaborating on the attributes of these principles of composition.
