\chapter{Conclusion}
\label{ch:conclusion}

This work has attempted to represent as a starting point of metaphysics, the phenomena of change found in the natural world, in order to argue the position that an irreducible composition of principles is found in everything that undergoes change, and furthermore, this kind of composition is hylemorphic for all material things.

The argument was developed in three stages. The first stage showed that there must be at least a principle of persistence and a principle of change in the human intellect in order to affirm or deny the the thesis of the reality change. The second stage related the principles of persistence and change to at least living organisms through Aristotle's theory of substance and accidents. The final stage expanded the theory of composition to all material things, arguing that in order to have instances of a kind, hylemorphic composition appears to fulfil this requirement.

The core of the hylemorphic position is that everything that changes is a kind (or nature, or essence, or substance). This position at least implies that everything has a corresponding range of potential changes it can undergo, whereby the range is bound by, or coordinated by, the kind of thing that it is. Furthermore, the kind of a thing is distinct from its individuating matter. What undergoes change is the composite, not the kind nor the individuating matter.

The position is not one which lends itself to being readily accepted. A first conclusion of the position, though not emphasised in this work, is that there must be an ultimate bearer of change, pure act, as concluded by Aristotle. This position is affirmed by Aquinas in his Five Ways, where Aquinas concludes that this is what we understand to be God. (\acrshort{aquinas-summa-th}, I.2.3) Therefore, the position entails a central tenet of classical monotheism. Secondly, the position is, at least in appearance, counter-scientific.
%%
If only composites undergo change, rather than kinds, and if speciation just is the differing in kind, the position does not obviously admit mutation of species by efficient causes acting on composites.
%% This implies that for evolutionary theories to be tenable, causes would need to be identified which could plausibly account for changes in essence.

On the other hand, one of the persuasive aspects of the hylemophic theory lies in its capacity to ground principles of unity and activity which carry over into human nature. These principles are indispensable for responding to the mind-body problem. Instead of requiring an altogether novel principle in order to account for the unity of personal actions, thereby introducing an additional substance which exactly coincides spatially with the body, as \textcite[][9]{Lowe2006-LOWNSD} does, substantial form, which is pervasive throughout all material things, is capable of fulfilling this role. The hylemorphic theory maintains the human person as a single substance, ruling out the possibility of a `bionic' body being bound to a personal substance. \parencite[][9]{Lowe2006-LOWNSD}

Both substance and individuation are essential aspects of the hylemorphic theory, but both appear to be in need of further clarification. Is substantial form necessarily unchanging or else what kind of causes could bring about change in it? Secondly, how is subsistent form individuated upon corruption of the composite of the human person? Finally, relating substance to individuation, is the notion of energy sufficiently open so as to be a bridge between the informing substantial form and the informed individuated matter? These three questions touch upon some of the difficulties presented in the development of the thesis of hylemorphic composition and would be suitable topics for further investigation.
