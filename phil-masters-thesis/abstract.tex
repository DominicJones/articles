\newpage

\begin{abstract}
The problem of how something could persist through change lead Aristotle to formulate his theory of hylemorphic composition, whereby every material thing is composed of the irreducible principles of matter and form. This work aims to affirm Aristotle's position by considering the reality of change, the necessity of a bearer of change and finally, the individuation of bearers of change.

Hylemorphic composition, being a metaphysical theory, touches upon many areas of philosophical enquiry. One area is the mind-body problem. In tandem with developing Aristotle's theory, the suitability of hylemorphic composition as a response to the mind-body problem is presented. The secondary aim of this work is to show that hylemoprhic composition offers a credible response.
\end{abstract}
