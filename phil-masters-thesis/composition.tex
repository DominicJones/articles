\chapter{Composition}
\label{ch:composition}

In Chapter 3, the case was made for affirming the reality of a single bearer of change, the substance, which supports and provides the coordinated unity among accidental changes of a subject. Such subjects at least include living organisms. Both the substance and the accidents in a given subject were argued to be irreducibly distinct metaphysical principles, hierarchically ordered according to accidents being dependent on substance for their existence. This position broadened the conclusion from Chapter 2, where at least the human intellect, a faculty capable of change, was shown to be in some way dependent upon a bearer of change, therefore denying the standard positions of both Heraclitus and Parmenides, that everything changes and nothing changes, respectively. So far, then, composition in change has been shown to be a metaphysical reality, though not necessarily extending beyond living organisms and not necessarily hylemorphic in nature.

At this stage the two strong contemporary positions which are compatible with the argument are non-Cartesian substance dualism (NCSD) of Lowe and hylemorphic composition of Oderberg and Feser. Property dualism, panspychism and physicalism, due to their failure to provide any irreducible principle that could somehow account for persistence through change are considered to be in principle incompatible with the conclusions developed already.

The third and final stage of developing the argument for composition in change is to affirm its reality in non-living things, thus affirming that composition is a feature of at least all material things. This step aims to rule out NCSD, leaving hylemorphic composition as the most credible metaphysical account of the phenomena of change in the physical world.

The argument for affirming the compositional nature of non-living things will not further elaborate considerations about change but rather it will focus on individuation of kinds of things. This approach puts to one side whether or not real essentialism is pervasive throughout all material things, not just living things, since arguing this position becomes less persuasive as materially simpler things are considered. Instead, the position that real essentialism does pervade all material things will simply be asserted, insofar as all kinds of things at least tend towards some range of outcomes rather than others, indicating some minimal degree of operations and therefore some minimal degree of being bearers of change.

With respect to the mind-body problem, concluding in favour of the position of hylemorphic composition, in contrast to NCSD, entails that the fundamental principle which is the bearer of identity, or `self' to use Lowe's terminology, is not an novel principle for beings which have a `self' but rather is a principle found in all changeable things, but bearing powers, or faculties, proper to the kinds of things that they are. From the mind-body point of view alone, hylemorphic composition could readily be seen as a mere difference in emphasis compared to NCSD; the former adds faculties to substantial form whereas the latter adds substantial unity to actions directed by a principle of self. Deciding between the two positions solely from the perspective of the mind-body problem would be very difficult. Widening the metaphysical scope, however, to examine the experience of change breaks the deadlock, as argued here, because the the single theory of hylemorphic composition is adequate for both change and the mind-body problem. NCSD, on the other hand, does not appear to have the necessary irreducible principles to accommodate change, at least for organisms that lack selves.


\section{Individuation}

Change has been the starting point of the enquiry into composition. After this, substance and accidents have been described to develop the notion of persistence through change in living things. The task now is to determine whether or not there are further principles which need to be posited in order to account for individuation of things of the same kind, and if so, how do they relate to the principles established already.

At first, there is no obvious reason why further principles need to be posited to account for individuation, over and above the substance and accidents categories already presented. Substance is described as `not present in', implying a notion of something self-contained and individuated. Furthermore, quantity is presented as one of the accidents, something present in the subject. It is quantity that divides parts from whole; without quantity there can be no parts and therefore no extension. This, however, is not considered adequate by Aristotle and his commentators, principally Aquinas: `accidents present in substance' is a distinction which is too general to resolve material individuation. (\cite[][103]{dominguez1991metaphysics}; \cite[][11]{klima-aquinas-contemporary})

The inadequacy of the Categories as can be seen by considering the essence of things. Rational animal is the logical essence of humanity; it presents the proximate genus and specific difference.
%% But it does not present human personhood is its logical essence.
But no human person is only its logical essence.
%%
To be a particular human person is to belong to that kind \emph{and} to be materially existent.\footnote{`[T]he definition [of what something is] expresses the \emph{essence}. The essence is not the form. Where there is a form-matter compound, the essence is expressed by the definition of that compound in terms of its form and matter \emph{in combination}. [...] This might make one wonder how an Aristotelian can even separate form from matter so as to be able to say \emph{what} the formal component is. The answer is that in a sense he cannot. It is not as though the form can be held up for inspection independently of the matter and then given its \emph{own} definition.' \parencite[][167--68]{Oderberg2014-ODEIFS}} The physical essence includes, therefore, that it is a materially extended rational animal. However, to posit the accident of quantity as that which individuates the essence would conflate the sense of accidents being `present in' substance, and the sense of substance being an individual particular. Accidents simply make for the possibility of change in something but are not themselves the bearers of those changes.

To propose that the accident of quantity is that by which something is an individual would be to make a material subject dependent on an accidental category, creating a circular dependency. In Chapter 2, change and bearer of change were shown not to be equally fundamental principles, but on the contrary priority must be given to the bearer of change over any given actual change. Being quantified is dependent on being something, rather like the priority of there being a cat of four kilograms not four kilograms of cat-like quantity; the weight can change without corruption of the substance.

Quantity gives a subject a specific, or determined, extension at any given point in time. What quantity does not provide is the principle by which quantity is received in a subject in the first place. A principle, therefore, that determines that something has quantity is needed. This principle needs to be such that the original matter when the subject is composed does not need to persist throughout the persistence of the substance, but, nevertheless, some quantity must still make up the subject at any given time. The accident of quantity provides extension, here and now, and its variation, but something needs to account for the fact of being quantified in the first place. To illustrate this difference: in order to possess the concept of triangularity, a recipient must be capable of educing from examples of triangles the concept of triangularity. A person merely having the concept of triangularity \emph{would not also} explain the capacity to receive the concept in the first place. A similar arrangement appears to be the case for individuated kinds of things: quantity, like triangularity, is received, but not just by anything. Cats and dogs do not receive the concept of triangularity, just like angels do not receive quantity; rather, only those kinds which have the capacity to receive quantity may do so.

%%

Some principle for individuating the essence of a subject is needed which has the characteristic of neither being a substance nor an accident. The principle cannot be identical to the substance since the individuating matter can entirely change without the substance corrupting, and it cannot be identical to an accident since this would make substance dependent on an accident.

The principle of individuation must be indeterminate, since accidents already are determinate with respect to degree, and essence is determinate with respect to operations. A principle of the requisite kind could not be admitted by potency-act composition since there is no composition in a principle devoid of being anything particular.
%%
Potency, alone, however, could satisfy the characteristics of this required principle of individuation.
%%
Potency, as described earlier, has three principle characteristics. It cannot exist in its own right but rather it is present in act; it limits act to some particular kind; and finally, it is the principle of change.
%%
Of these three characteristics, it is the first which supplies the requirement of an individuating principle. The principle of individuation is the account of potency in its least differentiated manner, and represents the extreme opposite end of the ontological spectrum from pure act.
%%
This kind of principle is referred to as prime matter, a principle that is not obviously admitted by Aristotle, but is explicitly admitted by Aquinas.\footnote{Oderberg argues that prime matter is not the principle of individuation. Interpreting Aquinas, he argues that it is `matter \emph{possessing} indeterminate quantity'. Nevertheless, he remarks that there is the possibility that this definition `involves a circularity in the individuation of substance and accident'. \parencite[][]{Oderberg2002Individuation}}

It is tempting to conflate prime matter with matter, stuff or quantity, but this would be a mistake. Prime matter, as least differentiated potency, and therefore radically undetermined, is only a real principle but is in no way actual. The sense of it being real conveys that prime matter is causally relevant (unlike numbers or unicorns), whereas the sense of it not being actual conveys that it is only causally receptive (unlike an agent of change).
%%
As yet, it is not delimited or identifiable or quantified. Nevertheless, it is this principle by which the essence of a kind of thing is particularised in the natural world. Prime matter, however, must be composed with another principle to bring about real existence here and now, a principle that is actual to some degree. Given that the essence of natural things is described in the particular, already bearing the principle of individuation, that which receives prime matter and individuates essence requires its own term. `Substantial form' is typically used to describe the principle that receives prime matter and establishes the coordinating unity of the composition.

Aquinas offers an account of prime matter and substantial form, indicating why they are principles in themselves, and also principles of composition:
\begin{quoting}
We should note that prime matter, and even form, are neither generated nor corrupted, inasmuch as every generation is from something to something. That from which generation arises is matter; that to which it proceeds is form. If, therefore, matter and form were generated, there would have to be a matter of matter and a form of form ad infinitum. Hence, properly speaking, only composites are generated. (\acrshort{aquinas-principiis} 2.15)
\end{quoting}

The argument for the individuation of kinds into particular instances associates a principle of indetermination, prime matter, with a principle of determination, quantity, whereby quantitative determination is an act of the substantial form. This general structure of associated principles does, however, appear to be vulnerable to a critique when considering the individuation of `separated souls' upon death.
%%
The following two premises appears to lead to a conclusion which is incompatible with hylemorphic compostion:
\begin{enumerate}
\item the quality of intellectual character varies from person to person;
\item the subsistent faculties of the human person which are not intrinsically bound to matter are argued to persist upon death;\footnote{For the immateriality of the intellect, see \textcite[][219--28]{feser2006philosophy} and \textcite[][93--98]{Oderberg2005-ODEHD}.}
\item therefore quality, at least with respect to the intellect, is an individuating principle.
\end{enumerate}

If quality, with respect to the intellect, is a principle of individuation of post-mortem subsistent faculties then it should also be a principle of individuation in the single substance of the whole human person. This, however, would imply that there are two principles of individuation, introducing an apparent contradiction into the sense of unity of the single substance that hylemorphic composition insists upon. Comparing the post-mortem human soul as being in some way analogous with angels would be mistaken. Angelic kinds have no individuation; they are exhaustively their own kind. For angels, it is essence that individuates. \emph{Homo sapien}, however, does not individuate; rather, \emph{it} is individuated.

To respond to this dilemma, some way of showing that quality cannot be a principle of individuation is needed. This could be done by appealing to the earlier argument that, like quantity, quality cannot be a principle of individuation since it is already dependent on substance. This line of argument does not so clearly work, though, since post-mortem subsistent faculties are not substances, according to the hylemorphic position. The critique appears to have some weight and so is noted, though since hylemorphic composition does not necessarily entail the immortality of the human soul, the critique does not demand the reworking of the theory of individuation.

At this point, a full ontology is proposed that affirms the central claim of the thesis: potency-act composition describes change in general, substance-accidents composition describes natural kinds of things that undergo change whereby there is a single principle of unity, the bearer of change, and a multiplicity of possible kinds of changes, and finally, prime matter-substantial form describes the individuation of kinds of things. In each stage of compositional principles there is a narrowing of applicability: all reality falls under act or potency-act composition; all changeable reality falls under substance-accidents; and all material reality falls under prime matter-substantial form. Composition is found in all reality that undergoes change, and for material kinds of things, this kind of composition is argued to be hylemorphic.

The second part of this chapter will present and respond to the position of non-Cartesian substance dualism, since it represents the principle alternative to hylemorphic composition. This position will primarily be presented through the work of Howard Robinson, who, whilst not explicitly taking the position of NCSD, accepts the Aristotelian position of the immateriality of the intellect, but rejects pervasive Aristotelian hylemorphic composition, especially in non-living organisms. The explicit position of NCSD will be presented through the work of E. J. Lowe. Since Lowe, however, does not contrast his position with hylemorphic composition but rather the position of physicalism developed by David Papineau, this latter position will be presented to provide sufficient context.


\section{Physicalism}

Physicalism affirms that natural phenomena are ultimately accounted for in the details of the physical interactions of fundamental physical phenomena.  All physical effects are brought about by physical efficient causes, so there is no room for other causes without implausible systematic overdetermination. Papineau argues that there is a strong case for `closure under physics', and this serves as the foundation of the physicalist position. (\cite[][]{Papineau2001-PAPTRO}; \cite[][]{papineau2002thinking}) Robinson argues that closure under physics and hylemorphic composition cannot be reconciled. \parencite[][]{Robinson2014-ROBMHA-3}

For the physicalist, all material things behave according to physical laws, of which two are given special attention.
%%
%% For the physicalist, all physical effects are brought about by physical efficient causes, so there is no room for other causes without implausible systematic overdetermination.
%% , since to admit it as a cause would lead to an implausible systematic overdetermination of causes
According the laws of conservation of momentum and energy, the domain of possible causes that can influence material things is closed, thereby excluding metaphysical principles that cannot be subject to detection and measurement.
%%
One candidate for exclusion is substantial form. The hylemorphic position argues that substantial form is a real, immaterial principle, fully in every part of the matter it informs, and the cause of coordinated unity of the substance. But, positing such a principle, for the physicalist, is untenable, since causal influence is already closed off to non-detectable influences, so it is impossible for an immaterial principle to be a source or sink of momentum or energy to a material thing. For example, it is not possible, according to these laws, for some `coordinating unity' to coordinate the activity of raising one's arm because to trigger the event of raising one's arm by a coordinating unity would have to introduce a momentum or energy contribution in order to influence the bodily action. By introducing such a contribution, however, momentum or energy will not be conserved.
%%

Furthermore, science, especially physics, appears to reveal a `bottom up' description of nature, whereby the macroscopic structures and behaviours can be recast in terms of microscopic phenomena. This view appears to go against the hylemorphic position whereby the fullest sense of what something is lies at the substantial level, which may well be macroscopic, and it is the substantial form that properly governs, or at least constrains, lower activities, in somewhat of a `top down' fashion. For the physicalist, ice forms not because `ice form' comes along and turns water into ice but because at the atomic scale the hydrogen atoms lock together in a pattern that forms a crystal. What is seen at the larger scale is explained scientifically at the smaller scale. `Form', for the physicalist, is simply a pre-scientific account of nature.

Physicalism has become a plausible philosophical position since the mid twentieth century due to advances in the natural sciences describing conservation laws which appear to assert that any physical effect must have a sufficient physical cause. For Papineau, physicalism can be argued from three premises:
\begin{enumerate}
\item all physical effects are fully determined by law by a purely physical or prior history;
\item all mental occurrences have physical effects;
\item physical effects of mental causes are not all overdetermined.
\end{enumerate}
The conclusion he draws from these premises is that mental occurrences must be identical with physical occurrences. \parencite[][6]{Papineau2001-PAPTRO}

The causal closure argument has appeal. By its scientifically backed empirical claim, the argument provides a way of dismissing non-physical ontological principles, such as substantial form, since any causal influence from these principles would violate of the conservation laws. Whilst the argument does not deny the possible reality of non-physical principles, what it does do is remove such principles from contact with the natural world. Alternatively, it may permit metaphysical principles, but \emph{only as explanations} rather than as real. An example of this would be to consider the four causes of Aristotle as merely a way of describing a state of affairs for the sake of convenience, whilst acknowledging that the reality nothing more than conservatively governed physical activity.

The grounding for the first stage of the argument for physicalism appears, however, to overlook the implications of the second law of Thermodynamics. This law describes how closed systems will tend to a uniform energy state, a state of maximised entropy, in the absence of any work being done on the system. Considering this law against the reality of living organisms, how do such local low entropy states exist (namely, the living organisms), despite the universal empirical law stating that entropy will increase, but for external work done? The straightforward response to this objection is that the locally low entropy states persist because they do indeed consume energy to maintain such a state. Animals eat and this energy is absorbed and gives off heat and as such they too are net contributors to increasing the overall entropy. A second response is that local entropy reduction is argued to occur in nature anyway, so the phenomena of living things may not exhibit any special features. Hawking radiation emitted from a black hole is claimed, by Sir Roger Penrose, to be an example of this. Nevertheless, the death of a living thing is unlike the decay of a black hole; the former leads to a sharp rise in entropy without any clear-cut physical cause, whereas the black hole decays gradually \emph{because} it emits Hawking radiation.

It seems incumbent upon the physicalist to account for, within the domain of physical closure, the change from a coordinated unity in a state of relatively low entropy to an uncoordinated body of matter rapidly bringing about mutual decay and effective rise entropy.
%%
%% Examining death under the light of the claims of conservation laws is not presented here in order to deny such laws but to consider it as an entry point for challenging the first premise of the causal closure argument.
%%
Death has both an empirical aspect and a non-physical aspect. It is empirical insofar as being able to tell when something is alive and when it is dead. The flattened fox on the road is dead, the cow in the field is alive, and these are common-sense observations. The non-physical aspect is when the living organism actually dies. As Oderberg argues, this is a metaphysical change whereby the matter can no longer support the substantial form, as oppose to something that can be narrowed down to a specific change in a physical state of affairs. \parencite[][]{Oderberg2017Death} The moment of death lacks sufficient material identification, and this is not for want of trying to propose candidates.

Attempts to associate death with a particular organ failure, such as the brain, have counter-examples, as Oderberg discusses. Death, as he argues, is fundamentally qualitative in nature and is better understood as a process rather than an all or nothing point in time change, a process of mutilation (an increasing loss of coordinated unity) of members of the organism until the principle of life can no longer be sustained. It may still be maintained by the position of physicalism that a thorough account of death is simply pending further investigation, rather than death being in principle outside the enquiry of science. However, this position lacks the grounding of being able to posit a material hypothesis for death, being able to conceive of what death might be, in a material sense, in order to examine possible candidates for its cause. This is not to be confused with what contributes to the process of radical mutilation. Obviously, incidental mutilation contributes to this final change, but nevertheless, the activity of change is not what is being sought, rather loss of being of a particular nature, or the outcome of change, is the matter of concern.

%%

A second concern about the first stage of the the argument for physical closure is that necessity is implied by the conservation laws. Anscombe challenges the claim that there is a necessary connection concerning causation, that some event always follows such antecedents. A common position is that there is a relevant difference `if an effect in one case and a similar effect does not occur in an apparently similar case'.\footnote{Aristotle, Spinoza, Hobbes, Kant, Mill, as quoted by Anscombe, qualify their account of causation with necessity. \parencite[][]{Anscombe1993-CAD}}
However, she argues that what are necessary are laws of nature and from them we can obtain knowledge of effects from knowledge of causes, but such laws do not show us the causes as sources of the effects. Causation is the domain of the particular, whereas law is the domain of the exceptionless generalisation. Attempting to transition from the general exceptionless law to the particular whilst maintaining the quality of being exceptionless is an additional fact which, to her, lacks justification. When effects derive from causes, with examination we come to know what the sufficient, or normal, conditions are for the effects to follow. But sufficient or normal conditions is a vague notion: sufficient is very much like `enough', that there are enough conditions for the expected effects to come about, but such notions do not commit out of necessity that they do.

There is at least some common ground between what is observed at the scientific level and what is affirmed at the metaphysical level, following the hylemorphic account. What makes something transform either accidentally or substantially, is the efficient cause.
\begin{quoting}
According to our manner of knowing, the efficient cause is the first one we call cause: it is the one we usually refer to when talking about causes. [...] [I]t is the one that properly acts, and it acts by determining the material cause through the formal cause. The material and formal cause do not act: they cause each other. It is the efficient cause that acts. \parencite[][108--11]{de1981christian}
\end{quoting}
Not surprisingly, the scientific account of change is in keeping with the metaphysical account, but nevertheless, it is not the same account. The scientific account concerns itself with those aspects of being which can be measured, namely quantity, and those aspects that can be measured in relation to others, such as action, location and time. The scientific approach does indeed have a view on reality, but it is only certain accidental realities, not the whole scope of accidents, especially quality and relation, and certainly not substance.

The hylemorphic approach seems to agree with Anscombe's objection to necessity being an assumed condition of causality. How can necessity be drawn out of contingent beings in the first place? The principle of non-contradiction may be argued to be an exemplar: it is a fundamental notion about things in the world, that something cannot be and not be in the same manner. However, causality concerns \emph{transformation} of being, not principles of being. Causality follows from being but not vice versa, so for contingent beings, necessary causality cannot follow in principle.

In an indirect way, Anscombe is affirming the hylemorphic position that in the primary sense what is real is being, not becoming. In separating out the understanding of something being caused and something being determined into the former having taken place and the latter yet to happen, she argues that what is yet to happen cannot have the status of necessity attributed to it, as in having the initial conditions plus the principle of necessity in causes, entails something as being determined. This she rejects. In what sense can it be said that the path of a ball through a Galton board is determined? Attempting to work backwards from the final state of a ball in a particular bucket to how it arrived there from the initial state of it dropping into the grid of pegs rapidly becomes a futile task as compound error would swamp such a calculation. What can be accounted for rationally is that the ball did fall into a particular bucket and simulations on a computer can mimic the same activity. Instead, the assumption that causes can be traced back from actual events unambiguously, or rather, determinedly, is an illusion, Anscombe claims. Attempting to track upstream to the causes of events is asymmetric in the kind of knowledge received from advancing forward through events. The former is at best statistical whereas the latter can be described in the fashion of universal laws.

The approach of Anscombe to associate causation to particulars and determination to laws opens up some room to question the conclusion of the inadmissibility of causal overdetermination by the physicalist. That something may be overdetermined according to physical laws does not necessarily imply that it is `overcaused' in the particular things that generally behave according to such laws, particularly in the interaction between the mind and the body.
%%

A third response to physicalism is offered by Lowe, who argues that some form of ontological dualism is compatible with even the strongest forms of closure under physics. He argues that no matter which variation on the definition of the first premise is used (variations on: `all physical effects are fully determined by law by a purely physical or prior history'), the anti-dualism conclusion that `mental occurrences must be identical
with physical occurrences' cannot be affirmed. Too strong a first premise would render the third premise (physical effects of mental causes are not all overdetermined) redundant, thus leaving a two premise argument with the second (all mental occurrences have physical effects) agreed by dualists. Drawing the conclusion that some mental events are physical events from such an argument would not hold, however, since it would not be accepted by the dualist. Too weak a first premise would run the risk of it leading to invalid conclusions in arguments it is deployed. \parencite[][575]{Lowe2000-LOWCCP} From the point of view of the physical scientist,
\begin{quoting}
the mental event would be invisible to him [...], his explanation would be [...] incomplete and would falsely represent the occurrence of [a physical outcome] as being coincidental [with a mental antecedent]. \parencite[][581]{Lowe2000-LOWCCP}
\end{quoting}
Instead, Lowe's own suggestion is that
\begin{quoting}
The world of [mental and physical events being inter-causal] may in fact be our world. If it is our world, physical science can present us with the semblance of a complete explanation of our bodily movements, and yet it will leave something out, giving our bodily movements the appearance of events in our brains and nervous systems. But isn't that precisely what current physical science does appear to do? As it traces back the maze of antecedent neural events, it seems to lose sight of any unifying factor explaining why those apparently independent causal chains of neural events should have converged upon the bodily movements in question. In short, it leaves us with a `binding' problem associated with conscious perceptual experience. \parencite[][581]{Lowe2000-LOWCCP}
\end{quoting}

Lowe also considers the nature of energy too opaque to delimit the manner by which contributions or losses are made in its overall conservation. If energy is a physical quantity then the closure argument becomes merely verbal since conservation of energy is appealed to as grounding the first premise. Given the openness of the ontological nature of energy, kinds of energy could be postulated that are convertible into physical energy in accordance with conservation laws. \parencite[][571]{Lowe2000-LOWCCP}

Finally, Lowe argues that if there is no such thing as self, `a persisting entity an bearer of properties', then it is not clear how simple things, like moving one's arm, can be explained. Ordinarily, upon deciding to move one's arm, the arm moves. At the chemical/biological level, when tracing the outcome back to its causes, there is no correspondence between the single decision at some point in time and the single act of moving one's arm compared to the single outcome of the arm moving and the tracing back of physical contributors to it, which become more and more numerous and diverse the further back in the process we go. So, physically, there are many diverse and contributing inputs to the movement of my arm, but in intention there is only one. Therefore, there must be some immaterial principle which at least coordinates the body's movements.


\section{Top-down causation}

Although Lowe rejects a purely physicalist view, this rejection does not obviously extend to things that do not, for him, possess the dual principles of `self' and `body'. Where the line lies is not clear, perhaps including both sentient and rational animals or perhaps just the latter. Regardless, for non-sentient living things and all non-living things,  Lowe's position may reduce to that of physicalism. Robinson argues that the kind of position Lowe adopts is preferable than a hylemorphic view, even if the hylemorphic position can accommodate the requirements that Lowe sees as needing to be fulfilled for the self-body duality. The problem with hylemorphism, for Robinson, is that principles of substantial form and final causation (teleology) persist even in the absence of rationality.

How can closure under physics, an approach which seeks to explain the reality of things from fundamental components, be reconciled with hylemorphic composition, an ontological account of reality whereby parts are composed into a unified and coordinated whole by a `top down' principle, namely, substantial form? Robinson argues that a realist hylemorphism cannot be reconciled with closure under physics since:
\begin{enumerate}
\item all physical forces operate at the bottom level, so higher causes, such as structure or unity, are not ontological causes but explanatory schemes;
\item if high causes are in some way real, they are not part of the world they explain but only really part of the conceptual structure we apply;
\item the autonomy of structure points to a conceptualist interpretation of them.
\end{enumerate}
Robinson asserts that it is specifically the commitment to an irreducible teleology which prevents reconciliation of physical closure with a realist and substantive hylemorphism. To illustrate his scepticism about substantial form, he compares a the causal cohesion of an oak tree and the dynamic interaction of swarming insects and argues that the manifest particulate nature of the latter is merely a difference of complexity and degree compared to the former. `One might put it this way: an object does not need to be something over and above what constitutes it to be real; if its constituents are real, so ipso facto it is.' \parencite[][208]{Robinson2014-ROBMHA-3}

Robinson's primary reason for rejecting the formal and final causes is that he thinks that they are only worth positing if they make a difference to the distribution and motion of matter, over and above what one would expect from the efficient and material causes. But to do this is inconsistent with modern science.
%%
Modern science, he argues, tells us that the physical world is causally closed; that all physical effects can be fully explained by prior, efficient physical causes. Therefore, formal and final causes, as described by Aristotle, are redundant.

If the hylemorphic position were affirmed, substantial form would dominate the understanding of things that have no obvious sense of substance. Robinson makes use of the phrase `distribution and motion of matter' in order to emphasise what he regards as the pathological case for hylemoprhic composition. Hylemorphic composition appears to wish to assert form on everything, giving everything some `top-down' power which governs how it behaves. This position, however, is simply not the view which the sciences, especially physics, conveys. The search for scientific explanation tends to advance by understanding the characteristics of components, rather than wider environments.

Hylomorphism, Robinson contends, simply does not fit nature as is now understood. The incompatibility of the four causes with closure under physics is that there is no such separation between the material and efficient causes being the domain of the natural sciences and the totality of the four causes, material and efficient, along with formal and final, for the domain of everything else, such as the mind-body problem. This incapacity to partition the causes is due to affirming hylemorphism, since it brings form into the ontological account of everything, since it permeates the whole of material nature.
%%
If form does play a role, and so vindicate the Aristotelian position, Robinson insists that `[its] relevance here can only be that form plays an essential role in making the matter do what it does', as a top-down controlling influence on matter rather than something that merely explains why there is an organisation present in the first place.
\begin{quoting}
There is a view current that there exists entities that are real but not fundamental, and perhaps substantial form falls into this category. But in so far as these entities make no difference to the location or movement of matter I do not see how they can occupy the foundational position that Aristotle attributes to substantial form. [...] Aristotle's substantial forms or essences [...] supposedly possess a metaphysical unity no plausible account of which seems to be available. [...] [I]t is hard to see what the reality of such forms consist if they are not just convenient explanatory categories and yet do not count as efficient causes. \parencite[][5]{Robinson2018-talk}
\end{quoting}

Robinson considers modern interpretations of substantial form, principally the `rescue act [of] understanding [it] as structure or organisation'. But he rejects this interpretation since structure is a by-product of the interaction of parts rather than the cause of unity. For example,  the path a river takes is not simply a consequence of the dynamic structures in the flow of the water, but also of the relatively static structure river banks that bound its course. Since the two structures do interact and influence each other, structure cannot be understood in an analogous way to substantial form, namely, a single coordinating principle.
On this point Oderberg agrees with Robinson that `structures in general are too prolific and overlapping to do the job of individuating objects that form is made to do'. \parencite[][7]{Robinson2018-talk}
%%
\begin{quoting}
Form has an irreducibly qualitative aspect, \emph{supplementing its quantitative aspects} which can usefully be called structure, or better still, structural features. Taken together ---taken holistically---  we get a picture of the form of a substance. \parencite[][178]{Oderberg2014-ODEIFS}\footnote{Emphasis added.}
\end{quoting}
%%

Feser appears to undercut the central issue Robinson may have with respect to hylemorphic composition by arguing that hylemorphic composition of matter and form makes, at the level of mere distribution and motion of matter, a minimal and scientifically non-invasive claim.
\begin{quoting}
Note that any determining, actualising pattern counts as a `form' [...]. Being blue, being hot, being soft, etc. are all forms in the relevant sense. Note also that `matter' is not meant here in the same sense in which it is used in modern science --- though hylemorphism is not in competition with modern science, just as [...] the notion of active potencies or causal powers is not in competition with modern science. \parencite[][161]{feser2014scholastic}
\end{quoting}
The argument he puts forward is that matter understood metaphysically and matter understood scientifically are not being placed at the same level of reality, but rather being placed hierarchically. So, whilst talk of causes and hylemorphism at the scientific level may seem unjustified, at the metaphyiscal level it is necessary.
\begin{quoting}
Whatever chemists tell us about the chemistry of ink, and whatever physicists tell us about the nature of matter more generally, change presupposes `matter' in the sense of a determinable substratum of potency. For the purposes of science, that, like the notion of a causal power, is ontologically minimally informative. \parencite[][161]{feser2014scholastic}
\end{quoting}

Robinson simply asks what way, causally, does form have anything to do with the distribution and motion of matter? Surely it is enough to claim that there is something, called matter, whereby disturbances in one portion of it at least locally effect other portions in a general, describable way, and that the fact the this happens is termed efficient causation.
%%
Form, according to Feser, is an `intrinsic principle by which a thing exhibits whatever permanence, perfection, and identity'. (\citeyear[][162]{feser2014scholastic}) But this is regarded as irrelevant by Robinson, since the mere distribution and motion of matter does not have any obvious sense of permanence, perfection or identity.

If hylemorphic accounts of even the simplest scenarios of mere distribution and motion of matter must be appealed to, then this necessarily rules out any bottom-up description of matter, which is the prevailing description from the natural sciences. So, is hylemorphism really top-down, in so far as any account of matter must make an appeal to some causal reality which is outside the domain that falls under examination of the natural sciences?



Consider the example of radioactive decay of an atomic nucleus: particles or radiation are emitted from an unstable nucleus in a stochastic process, i.e. one that is random but can be analysed statistically. Such a scenario fulfils Robinson's state of affairs under examination, the distribution and motion of matter. Can the very description, however, be arrived at in the absence of at least an implied ontological form and finality? Words that need grounding are `decay', `unstable' and `random', whilst avoiding appeal to anything ontologically beyond matter and efficient change, if Robinson's position is to hold force.

To decay is to break down from something that has greater being (or unity) to some things that, therefore, have a lesser degree of being. To state this of any decaying thing, though, is to acknowledge the initial undecayed state had, in some minimal sense, a unity. Likewise, to be unstable indicates activity, and in this case, towards reaching stability, conveying, again at least minimally, a directedness toward some end, in this case a stable isotope. Finally, something is random, not in its own right but relative to some activity that is systematic or regular. If there is randomness as such then efficient causality is incapable of accounting for its existence, implying there is some other irreducible principle in act. It is not obvious how the other classical causes could account for the phenomena of random activity, but nevertheless, positing randomness appears to not fit into a mere material-efficient ontology.\footnote{Aristotle appears to characterise randomness as the privation of an expected tendency. (\acrshort{aristotle-phys} II, 8)}

Following the themes of stability and randomness, a readily reproducible mechanical problem, the double pendulum, provides a second example. By introducing a simple modification to a regular pendulum of adding a mid-point pivot, the behaviour of its motion changes from non-linear to chaotic. What starts out as a problem with one degree of freedom (the regular pendulum) having readily predictable behaviour relative to the initial angle the swing of the pendulum starts at, a radical change takes place with simply introducing an extra degree of freedom, such that now the behaviour of the pendulum becomes very difficult to predict from the initial angle of its swing. So, if efficient causality is the exclusive account of activity, how does anything but trivial unities behave in stable ways? This simple example highlights an important characteristic broadly attributed to form which is the tendency toward some range of behaviours to the exclusion of others. How a double pendulum is put into motion is trivial, but how it behaves is intuitively inexplicable. The behaviours may still be argued to be merely `efficient' but it opens the question of why that in nature, from the bottom-up, things are not chaotic but behave in regular ways?

The folding of protein molecules, as a final example, takes to the extreme something having many degrees of freedom yet displays a unified and regular behaviour. What is it about this chain of amino acids with thousands of degrees of freedom that is the cause of it folding to its lowest energy state non-iteratively? Arguably, it just does, and that is efficient causality. But what is pulling the activity along? The scientific answer may be that it is an energy gradient which needs to be minimised. But it is not obvious as to whether or not gradients are merely explanatory or ontologically real.

Robinson may argue that gradients are merely explanatory, an abstraction that sits on top of whatever is actually there. This may indeed be true, but I argue that if it is, it is not obviously true. Almost all scientific simulation software \emph{predicts} natural phenomena through differential equations (equations that involve, fundamentally, gradients). It is gradients that `drive' the motions of fluid dynamics and solid mechanics. A gradient at some point is the spatial variation of some property in the vicinity of that point, where `vicinity' is merely a location distinct from the location at which the gradient is at.

Gradient, it seems, must affirm the principles of identity and relation in order to be real, since it is marking a difference between one location and another (identity of locations) and in some way recognises that that states in these locations are mutually `visible' (identities in relation).
%%
For example, when a metal rod is heated at one end a temperature gradient is introduced along the rod. Given enough time, and the right conditions, the the temperature of the whole rod will be raised to this new temperature, thereby removing any temperature gradient in the rod. If this process was modelled mathematically, the gradient would be said to `cause' the change of temperature throughout the rod. On the other hand, from Robinson's kind of perspective, no such causal reality of gradient exists, only its explanatory reality. What is not clear about Robinson's position is that, even if gradients are only explanatory, his position still seems to need to invoke some kind of reality which has the features of bearing identity and having some relation among common kinds, at whatever granularity, so as to cause temperature change. To have these characteristics, though, just is to have the characteristics of substantial form (principle of unity), the accidents of quantity and relation, and, of course, to be material existent.

The examples of nuclear decay, double pendulum behaviour and gradient minimisation seem to show that a notion of `distribution and motion of matter' \emph{first} needs to affirm the reality of hylemorphic composition in order to identify the tendencies of such distributions and motions. To say of some kinds of distribution and motion of matter that they have no substantial forms may in fact be in complete agreement with the hylemorphist, if those kinds were like the example of the \emph{chaotic} behaviour of a double pendulum, \emph{an artefact} of two beams pinned together. But such an example is manifestly different from a decaying nucleus or a folding protein \emph{because} of their regular and stable tendencies, despite them being structurally complex.

It appears that Robinson's view of matter and efficient causation is scientific rather than metaphysical, therefore what has been presented here are arguments of a more scientific bias, though they do attempt to point to metaphysical implications. It would be incumbent on Robinson to account for the metaphysical possibility of change to strengthen his position of rejecting hylemorphism, since he relies on it to frame his minimal problem of the distribution and motion of matter, whereby nothing ontologically irreducible is appealed to beyond matter and efficient activity. No such argument is offered by Robinson, whereas an Aristotelian argument is offered that shows matter and form must be appealed to as irreducible principles of substance.
