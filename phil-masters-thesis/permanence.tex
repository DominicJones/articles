%% \footnote{Accidental is the antonym of essential only in the logical order. In the ontological order some faculty could be accidental but also essential, like the capacity to abstract in Homo Sapiens.}

\chapter{Bearer of change}
\label{ch:permanence}

The previous chapter argued that change is real, and that in order for a change to take place there first needs to be a `something' that can undergo change. The principles of potency and act were introduced as a minimum requirement for the distinction between the possibility of being something else, and the actuality of being something, here and now.

This chapter aims to expound the nature of these two principles and their relation to each other in order to make possible the position that there is persistence through change. Whilst potency and act justify the very existence of change, the principles of potency and act do not obviously answer the question of what makes something recognisably the same thing through change, like a sapling growing into a tree.

There are two main views about this question. The first view of change I shall term as `atomic': change is real but from one change to the next there is no persisting essence.\footnote{The `atomic' terminology is taken from \textcite[][336]{pinckaers1995sources}, where it is applied to the distinction between freedom of indifference and freedom for excellence. Freedom of indifference, atomic moral action, consists wholly in a choice between contraries and separated from all the actions preceding it or following it. Freedom for excellence, on the other hand, consists in habitually ordering acts with the perspective of the final end of the human person, happiness.} The second view is that which Oderberg calls `real essentialism': there is a persisting essence that grounds identity through change.\footnote{`Real essentialism defends the metaphysical position that everything in the world has an essence or nature that fixes its identity'. \parencite[][]{Oderberg2007-ODERE-2}} This chapter considers which view we take, having adopted the principles of potency and act. An atomic view was rejected in the previous chapter insofar as a person wishing to hold it would in the same act undermine the position being true for him. However, it does not follow from that argument alone that an atomic view of change is not true of non-rational beings.

The position of real essentialism will be defended and I will attempt to show that it is `act' that grounds all intermediary bearers of change.
This will be defended by arguing that persistence through change has a hierarchical dependency on some terminating principle that is absolutely unchanging. If, on the other hand, an atomic view were indeed shown to be pervasive, either the exception for human minds would need to be maintained, or else the argument of the last chapter would need to be revisited in order to see if it goes wrong somewhere.

\section{Hierarchy of categories}

Aristotle proposed a collection of distinctions that could be applied to things to characterise them. Following the sentence structure whereby a subject is described by a predicate, Aristotle built up a collection of ten predicates, or categories, that he considered true aspects of things found in the world. (\acrshort{aristotle-cat} 1b25--2a4) For example, `Felix is a cat', and `Felix weighs four kilograms' presents two of these categories. The first is what Aristotle calls a substance, `cat' in this case, and is the name than distinguishes what kind of thing the subject is. In taxonomical classification this would be the full name of the proximate genus and its specific difference (\emph{Felis catus}, say), thereby marking out a particular species. The second is its quantity. Other categories were also described by Aristotle, such as quality and relation, but these first two suffice for examining the credibility of there being a bearer of change which is also a bearer of essence.

An important insight in the Categories is that we can distinguish predicates for the sake of enquiring about a subject under some aspect or other.
Designing cat flaps would be concern quantity, such as the average size of cats, but not quality, such as their propensity to hunt mice. A second insight of the categories is their hierarchy. Aristotle places the substance category first, and this is followed by nine categories, referred to as accidents. The distinction he draws between the substance and accidents of a particular kind of thing (primary substance and accidents) is that the latter are only ever `present in' the former.\footnote{A similar distinction is be made between the substance and accidents of universals (or secondary substance and accidents), as discussed in \textcite[][]{sep-aristotle-categories}.} The quality `black' is present in `cat', for example.  This ordering of categories conveys that not all distinctions are equal in relevance from the perspective of the persisting subject.
\begin{quoting}
It is a distinctive mark of substance, that, while remaining numerically one and the same, it is capable of admitting contrary qualities, the modification taking place through a change in the substance itself. (\acrshort{aristotle-cat} 4b17--19)\footnote{The `taking place through a change in the substance itself' is interpreted here as meaning `change in the composite' rather than change in the bearer of change, since `substance remain[s] numerically one and the \emph{same}'.}
\end{quoting}

If a hierarchy of categories were not to exist, whereby there were no such thing as one which was the bearer of others, then it is conceivable that an atomic view of change could prevail, with the only defensible exception being the human mind, as argued in the previous chapter. Such a position would entail that one could hold either that there is a cat weighing four kilograms or that there are four kilograms of cat-like quantity, since at any given moment one would have to be impartial toward all categories, be it substance or quantity, etc.
%% This would be the case if there is some objective sense in which it is a cat that has the quality of weighing four kilograms, rather than a four kilogram thing that has the quality of being catish.
%%
In order to defend Aristotle's hierarchy of categories, some way of showing why one category must be the bearer of other categories needs to be demonstrated, showing how at least one predicate can be said to be a greater principle of persistence than another.

\subsection{Immanent activity}

Common sense suggests that Felix, a now mature four-kilogram cat, was once only one kilogram but, nevertheless, is still the same cat. Quantity, then, appears to be a relatively independent category. Indeed, quantity may change continuously, albeit to a relatively small degree, without causing the corruption of the kind of thing that bears the quantity. But, if all categories were like the category of quantity, admitting continuous, and possibly wide ranging, change, the notion of `bearer of change' would have very little meaning, nothing more than that offered by the principles of potency and act developed earlier.

To argue that there exist bearers of change in an objective, fundamental sense, it is necessary to explain why cats vary in weight rather than weight varying in cattishness.
That is, it is necessary to explain why this very object can survive a change in weight, if it remains a cat, but cannot survive a change in whether or not it is a cat, so long as it remains the same weight.
To answer this question, the transition from exhibiting immanent and transient activity to exhibiting only transient activity will be considered. Is the change from living to dead a change which can only be accounted for by a loss in an irreducible principle, its bearer of change?
\begin{quoting}
Living things, unlike non-living things, exercise \emph{immanent} causation: this is a kind of causation that begins \emph{with} the agent and terminates \emph{in} the agent for the sake \emph{of} the agent. \emph{Transient} causation, on the other hand, is the causation of one thing or event (or state, process, etc.) by another where the effect terminates in the former. All exercises of immanent causation involve transient causal relations as effects and/or instruments. \parencite[][180]{Oderberg2007-ODERE-2}
\end{quoting}

Exemplars of things that appear to possess an irreducible bearer of change are living things. At least in most living things it is generally held that there is some sense of an identity that persists despite near total change in quantity during their lifespans, or whereby memories and concepts are retained, or whereby certain ends are sought. But, is it the case that for each thing there is just one bearer of change? Two characteristics point towards an expected hierarchical view. The first is that change of quantity is partly reversible, the weight of Felix may vary non-trivially throughout its adult life. But once dead, there is no reversible process. Things change, but the example of death suggests that the kind of changes something could undergo, at least living things, are not all changes in a single sense. Some changes, like quantity appear to be less important with respect to the integrity of the thing itself compared with change from being the possessor of immanent activity to exclusively transient activity.

\begin{quoting}
Substances undergo changes while remaining the same in their being. It is accidents that come and go. The substance has potentially many accidents. Substance is to accidents what potency is to act, and as act perfects potency, so accidents perfect the substance. We also observe that substances change not only by addition or subtraction, but by becoming other substances. Substances do not change into any sort but into very specific substances. Before a substance changes into another, it has the potentiality to become precisely that one. A substance has not only the potentiality to receive and lose accidents, but also the potentiality to become other specific substances: two types of potentialities, one at the level of the accidents, and the other at the level of substance itself.
\parencite[][49]{de1981christian}
\end{quoting}

\subsection{Classification}

The task of classifying Felix the cat, or indeed anything else, is principally a task of going from the observation of attributes, like the colour of its fur, to seeking divisions among things that are seemingly pick out essential characteristics, like whether or not it is a living thing. This kind of enquiry seeks to recognise two important matters. The first matter is the difficulty in perceiving what is a readily changeable attribute from what is `standing under' it as the support or grounding that permits the changeableness whilst retaining the stability and uniqueness of the thing itself. The second is the broader matter of what, ultimately, is being classified.

For the first of these two matters, we could say that a cat is a vertebrate, which is an animal, which is a living thing. But does, for example, `is a vertebrate' divide along an essential line, a characteristic which is stable and must persist for the thing to remain what it is, or is it an arbitrary division that groups creatures eclectically, having no further likeness than the mere fact of having a spinal column? The task of classification suffers the difficulty of moving from perceived attributes to essential characteristics and appears to reflect the underlying philosophical difficulty of recognising what is essential from what is non-essential.

The search for the identity of something then, could be considered like moving inward through concentric circles of attributes, from those that are more peripheral to those that are more fundamental.
\begin{quoting}
There seems to be a hierarchy of attributes to which we attach relative importance in grasping a thing's identity [...] from the periphery where certain attributes [...] have fairly transitory importance, towards the center where [...] characteristic function assume dominance. \parencite[][79]{Oderberg2005-ODEHD}
\end{quoting}
All are real attributes but arriving at the centre we come to know that which essentially distinguishes and so identifies what something is. Returning to inherent tendencies, a similar picture should be expected. For substances with the highest natural capacities it would be expected that they have more readily identifiable principle of inherent tendencies and a spectrum of subordinate inherent tendencies given that such substances would not only possess the highest natural capacities but a whole host of subordinate capacities.

Justifying the notion of such hierarchies is simply achieved from defining a superior capacity as being one whereby it entails an inferior capacity if the former entails the latter but not vice-versa. For example, sentience is superior to nutrition as the sentience depends on nutrition, but nutrition does not depend on sentience.
\begin{quoting}
F-type capacities are superior to G-type capacities just in case the former entail the latter but not vice versa. [...] [T]he nature of a thing is defined in terms of its highest capacities.
\parencite[][88]{Oderberg2005-ODEHD}
\end{quoting}

Allied with this rule for determining dependency in a living organism, a case can be made for preferring an essentialist view of change as opposed to an atomic view. If there is indeed an objective hierarchy in living things, there is at least some principle that must be more stable than another in them. Classifying things then becomes a task principally concerned with identifying the most dependent capacities in living things, since these will reveal the differences among like things.

The movement from the least dependent to the most dependent capacities in the hierarchy of a living thing appears to be a movement from what is more changeable towards what is more stable. At the species level, a cat's fur may malt or the cat may lose a limb, or undergo many kinds of fine grain changes, but still be the same cat, a \emph{Felis catus}. Changes that fall below the granularity of the species, however, are simply no longer relevant to the kind but only to the particular instances of the kind. This dividing line is not obvious though. Classification is seeking to construct hierarchy of dependences and place the most dependent as characteristic of a species. Seemingly trivial differences, however, are also hierarchically highly dependent.  A not so different cat, at least to the naked eye, is the \emph{Felis lybica}, with a slightly leaner build and slightly longer legs, and a few other pertinent features. Somehow these small differences are enough to classify it as a different species. At one end of the spectrum, if the essentialist position is to be robustly defended then some means to draw a line between what are the most dependent capacities \emph{of a stable kind} and what are merely variations of given capacities in a kind is wanted. Oderberg seems to acknowledge this difficulty but responds to the problem by pointing towards the individuation of an irreducible principle as being the true locus of what divides the essential from the non-essential, namely powers that are proper to the `substantial form'.
\begin{quoting}
In general, what matters are the congeries of powers, operations, activities, organization, structure, and function of the object, whether it be something as bare as shape in the case of the diachronic identity of a circle drawn on a piece of paper, or something as complex as character in the case of the identity of a relatively higher animal such as a dog. {...} Hence, \emph{it is Rover's special way of barking at dinner time that is of more relevance than his color} ---after all, he could have been swapped for a twin from the litter--- and it is his mournful mien when refused a walk in the park that is of more relevance than his enthusiasm for chasing postmen.
\parencite[][]{Oderberg2005-ODEHD}\footnote{Emphasis added.}
\end{quoting}

At the other end of the hierarchical spectrum there are the least dependent characteristics, those of greatest stability. A cat is a vertebrate, which is an animal, which is a living thing.\footnote{That a cat is not a vertebrate \emph{and} an animal \emph{and} living thing, and so being a cat could not survive despite no longer being a living thing, is argued by \textcite[][81--83]{Oderberg2005-ODEHD}.}
But what lies immediately beyond `living thing'? What is found at the highest taxonomical category; is it a kind which is still only \emph{relatively} stable or is it a kind which is permanent? Something, substance or being are terms that may be used to denote the wider taxonomy category above living thing, but none of them obviously convey whether or not permanence is an essential characteristic of it.

By extrapolation, the most general classification must have the greatest degree of permanence, as it can admit all possibilities of variations by definition. For this reason `substance' appears as a reasonable term for the most general category since it is never `present in' anything but rather the bearer of possibilities. Furthermore, `substance' reflects the sense of stability expected from ascending to the highest level of classification; what readily changes are the accidents of things, like the cat gaining weight, rather than substance becoming something entirely different, like the cat dying. But substances, at least when considered with respect to living things, do not appear to be permanent things. When a cat dies it ceases to be a cat.

Both the thesis that a substance (its substantial form) is something necessarily unchanging and its negation presents difficulties.
%% The status of whether or not substance should be understood as something necessarily unchanging does present difficulties if either the position is denied or affirmed.
If substance were necessarily unchanging, it would be impossible to have taxonomical division whereby changeable substance could be introduced since its alternative would have to be unchanging substance, but that is already assumed in the term already.
%%
Secondly, with respect to the theory of evolution of species by natural selection, this theory would be ruled out since the bearer of change is required to change. The macro-evolution of a new \emph{species} develops from some pool of existing species, mutations and circumstances of habitat, rather than merely changes in the accidents present in the bearer of change. This is unlike the micro-evolution of a species which adapts to a certain environment, such as dandelions flowering with a short stem in grassy areas which are regularly mowed.
%%
The position of substance being unchanging does appear to have support in both Aristotle and Aquinas. For Aristotle, matter comes from matter and form comes from form; what is generated are composites. The argument being made seems to be that form, i.e. substantial form that is the bearer of change for material composites,  cannot be changed by some other principle, such as matter (or for Aquinas, prime matter), nor by its composite. The principles are irreducibly distinct and therefore are not subject to changes by each other. Even though cats die and wood is burnt to ash, this would not mean that the bearer of change has changed but rather that the composite has corrupted. Nevertheless, what happens to the bearer of change upon corruption of the composite is not altogether clear. Standard terminology states that the substantial form corrupts \emph{with} the composite, unless it possesses operations which are inherently independent of the compound material principle.

On the other hand, if substance were not necessarily unchanging, it would be reasonable, if one were working from the most general towards the most specific categories, to make the first division as `changeable substances' versus `unchanging substances', and then fill out the various divisions under `changeable substances'. This, however, would introduce a logical problem. The `unchanging substances' division would sit lower down in the taxonomy hierarchy, and therefore have greater affinity with species than `substance' since unchanging substance would be that one level closer to species than substance itself. Secondly, having the division unchanging and changeable subordinated to substance would necessarily give `unchanging substance' a greater sense of permanence than `substance' itself has, despite the latter being more fundamental and, presumably, more stable.

We might dismiss the above arguments for the unchangability or changability of substance on the grounds that taxonomy actually starts with what is known, like the various particular cat-like creatures, and then groupings are worked out. Speculatively starting at the top and hoping to get meaningful divisions that describe the essences of particular things is simply not how categories are developed. Alternatively, we might interpret the problem, not as a logical problem, but as proof that the hierarchy of greater and lesser generality does not map perfectly onto the hierarchy of greater and lesser stability after all.

However, dismissing the issue as an esoteric metaphysical problem that is not worthy of resolving would create problems for the application of the metaphysical theory in other areas, such as ethics. If it is the case that substances, or more precisely, substantial forms, are not necessarily unchanging then at least Natural law cannot be robustly defended since what may have been actions in accordance with the natural faculties of one genus of \emph{Homo} at one time may no longer be the case at a later time, among the same descendants, due to \emph{substantial} changes. The position of substantial form being necessarily unchanging has the authority of at least one broad philosophical school of thought, namely, the Aristotelian-Thomistic, but could be seen as pre-scientific. The alternative, where substance is not necessarily permanent, poses merely an apparent breach of the general rule that greater generality means greater stability, and so only a reduction in the elegance of the resulting theory.

The breach of accordance between generality and stability is not the only problem with taking the position that substance could be open to change. The fundamental problem with this position is that it provides no explanation of how change to composites effects changes to its principles, specifically to its bearer of change principle, its substantial form. Therefore, despite the appeal of substance being considered as not necessarily unchanging, this position is rejected in favour substance being necessarily unchanging. The remainder of the chapter will seek to develop the compositional view of natural things, showing the how the bearer of change must be immaterial principle, and therefore is not causally influenced by change, and that its capacity to undergo any change is due to its corresponding material principle.

\subsection{Coordinated unity}

Bringing together ideas developed in the above sections on immanent activity and classification, this final section on the hierarchy of the categories aims to show that at least for living things there is, in a real and irreducible sense, a single, immaterial and immanent coordinating principle of unity, which Aristotle refers to as primary substance. Furthermore, in the case of human beings, this principle, in addition to being distinguishable, is separable. Naturally, these claims considerably narrow down the range of compatable reponses to the mind-body problem.

The growth of a one-kilogram kitten to its mature adulthood of four kilograms involved consuming at least three kilograms of food. Three kilograms of non-cat became three kilograms of cat by being consumed. Cat-like operations that were not found in those three kilograms now manifest, as part of a whole, cat-like operations. Somehow, `catness' had the capacity to take on and inhere in something else without its own corruption, unlike the quantity of whatever the food was before being consumed. Substance, then, appears to have a kind of dominion over quantity. Upon eating a mouse, a cat does not become a cat-mouse, but does become a greater cat (as in, having more quantity). The consumer subjects what it consumes not merely to conform to itself but to be an increase in itself. But, suppose the mouse was eaten, and then sometime a little later regurgitated. Did the cat lose some of itself in the act of regurgitation? Surely not. Until the food is subsumed into the immanent operations of the substance, the food retains its own integrity. Looking from another perspective, no living thing would readily sustain the removal of a significant portion of its `quantity'; damage would be done.

Inhering in, then, has a certain strength of meaning, and it could be described as `inhering into a coordinated unity'. The regurgitated mouse was never assumed into the coordinated unity of the cat. From this, there appears to be some meaningful sense that quantity is present in (or is taken on by, or inheres in) substance, but not vice versa. On the other hand, for an artefact, like a computer, one part of it could be swapped for another without any adverse consequences to the functioning of it.

Perhaps discussing the substance of catness or the like is more a convenience of communication rather than a recognition of a metaphysical reality. Is it really the case that there a single principle that governs all activities, or are there many `principles' governing the various activities of living things?
The possibility of more than one principle is plausible: both breathing and heartbeat are necessary activities of distinct bodily systems in the life of a mammal. Nevertheless, even these activities need to be coordinated and kept in parity, and so instead, the nervous system could be plausibly considered as the single coordinating principle. Despite this, the nervous system is not an independent coordinating principle, and so not a principle in the metaphysical sense. There is an interdependence between many, if not all, bodily systems. One system coordinates all the signaling, another the supply of nutrition, and another the structural support, etc. Therefore, a plausible third answer is that there is, in fact, no coordinating principle but merely interdependent systems; the human person, for example, just is the systems that constitute it.

The consideration of the death of a living thing, I argue, provides a credible way for showing that there must be exactly one principle of coordination. There is a certain degree of mutilation that can be tolerated by an organism whilst not yielding to death. Blindness, deafness, loss of a limb, brain damage, etc, could all be described as mutilation. However, a local impairment to some particular bodily system, like a blood clot, could go on to corrupt the unity of the whole organism. When exactly the transition takes place from mutilation to corruption, however, seems not to fall under the enquiry of observation, indicating that there may be some kind of principle over and above the mere interdependence of bodily systems.

Attempting to associate death to some local change or other would undermine the very notion of being coordinated. Substantial change is across the whole substance, a loss in coordination of principles, rather than a change in some particular principle. This is not to imply that substantial change is somehow independent of local change, but rather that substantial change is not identical to a local change. Nevertheless, that the substantial change does take place becomes irrefutable. Loss of coordination of principles, then, rather than cessation of particular subordinated principles appears to be death as such. This does not rule out the possibility of a subordinated principle ceasing to operate being the main cause of death, but it is not death, rather only that which leads to it.

The process of death is somehow a local change effecting the metaphysical bond between it and a non-local coordinating principle. For a living thing, various kinds of mutilation, such as the loss of sight, hearing, limb, etc, diminish the capacity to move toward immanent ends, such as nutrition, growth, reproduction, etc, without destroying all immanent capacities. On the other hand, if the mutilation is to such a degree that no inherent ends remain achievable, there would no longer be a coordinated unity, since corresponding activity of that unity would be rendered impossible.

Attempting to positing a \emph{material} account of a non-local coordinating principle of unity would suffer from a vicious regress, since it in turn would require some further principle of unity due to it being extended. The alternative is that the principle of unity is immaterial. Being immaterial, the principle can coherently be argued to be wholly in the parts that it unifies. The limbs of a human body are not not limbs of a partial human being but of the human being in an unqualified way. Every coordinated part of the human body fully belongs to the human substance. If the principle of unity were not wholly in all the parts it unified, partial mutilation of a substance would not be possible. The loss of a limb would necessarily mean the irrecoverable loss of part of the principle of unity, therefore diminishing in some way the essence of substance.

Being an immaterial principle does not, however, imply that it necessarily continues to exist upon the corruption of the matter that it once informed and unified, though it does leave the possibility open. This possibility is employed with respect to the principle of unity in the human person. As noted in the introduction, it is the capacity for abstraction, an operation whose product, a concept, is immaterial, that distinguishes the immanent principle of rational beings from non-rational living things. The claim that the immanent principle of living things is immaterial is simply an exposition that at least living things are bearers of change. To ask `where is the bearer of change?' could only be answered by pointing to the living thing. But, what happens to the bearer of change when the living thing dies depends on the operations of the bearer of change. For rational bearers of change, there could be no such possibility of total mutilation, insofar as mutilation is to be understood as the impairment of a faculty from achieving its immanent operation. Damage to the eyes would mutilate vision, but what damage, or what change, could occur that would mutilate the power of abstraction to produce concepts? Seemingly no material change could produce such an effect. For non-rational bearers of change, however, every immanent operation could be impaired by impairing the least dependent operations, like nutrition.


\section{`Present in' }

In the previous chapter, any kind of change was shown to need some bearer of change, a principle that persists though a given change, and this principle was argued to be more fundamental than the change undergone. In this chapter the relationship between change and bearer of change has been examined with respect to substance and accidents of living things, whereby the bearer of change has been argued to be an irreducible immaterial principle. In both stages the notion of `present in' has been used in the two contexts: implicitly in the context of potency being dependent on act, and explicitly in the context of quantity (or quality, etc) being present in substance. If substance and accidents are terms that relate to potency and act for at least living things then the manner in which they relate needs to be established. Potency being dependent on act could be formulated as potency being present in act. Defending this same kind of relationship would offer parsimony of metaphysical principles, from principles of composition for all changeable reality to principles of composition for changeable living things. Therefore, how `present in' is to be understood needs to be examined to determine if there is indeed parsimony among the relationships of metaphysical principles so far presented. Since the explicit use of `present in' was with respect to substance and accidents, this will be taken as the primary sense. The task is to examine in what sense the notion of present in as understood for accidents present in substance can be applied to potency present in act. Possible interpretations could either equivocal, univocal or analogical, and these will be examined in turn.

For the equivocal sense of potency present in act, this would mean that potency being present in act is in no sense like accidents being present in substance, despite potency being dependent on act. In the case of accidents, their existence requires a bearer. Four kilograms does not exist in its own right but a four kilogram cat may do. Some accidents, moreover, whilst are only present in does not imply some accidents are not essential. A cat being four kilograms is accidental and is also not essential for being a cat; it could be three kilograms and still be a cat. On the other hand, the human capacity to produce concepts by the power of abstraction is a capacity present in humans and is essential; to not have the capacity whatsoever would not to be human, namely to be a Homo \emph{sapien}.  Accidents, then, are present in substance insofar as they at least facilitate its essential capacities. More typically, however, accidents facilitate a range of possibilities within a particular substance. Human being has a range of being male or female. The sex is present in human nature, and being sexed is essential for the kind, but being sexed has a possible range, male or female, for the particular. For potency to be present in act but no sense like accidents present in substance would mean that the dependency of potency on act is not a real dependency, i.e. potency can exist without act, and potency does not provide a range of possibilities for actualisation. Arguing an equivocal interpretation of potency present in act with respect to accidents present in substance would be tenuous. As argued already, potency does indeed provide a range of real possibilities, otherwise persistence through change would be inexplicable. Secondly, the ontological priority of act has also already been argued for, making the relationship between potency and act a real dependency. Therefore, it appears to be false that potency is in act in an equivocal sense to accidents being in substance.

The univocal sense of potency being present in act would mean that potency is present in act in just the same way that accidents are present in substance. This would be the ideal case for the argument from parsimony, providing a single kind of relationship, `present in', which is true for potency and act and also true for accidents and substance. If the relationship were to be the case it would imply that there were a variety of potencies and grades of relevance of potencies with respect to act, just like there are essential and non-essential accidents, and just like there are varieties of accidents, such as quantity, quality, relation, etc. Potency, however, has in no sense a relationship to act like essential accidents have to substance. If there is no such capacity of abstraction in a substance then the substance cannot be human. However, such a formula does not appear to apply to potency and act. `Is there a potency, x, such that if it were absent then some particular act could not be y?' is a nonsensical question as there is in no sense a multiplicity of kinds of act, rather there is only act multiplied in potencies. Therefore a univocal understanding also appears to be false.

The third way of interpreting `present in' is the analogical way. This interpretation implies that there is a significant similarity of meaning between the use of present in used in the context of act and potency and substance and accidents but that there are also differences which need expounding. With respect to similarities, both potency and accidents are dependent on act and substance, respectively, and both potency and accidents depend on act and substance for their existence, respectively. These two reasons establish the significant similarity of meaning between `present in' for both potency and act and accidents and substance. However, there are a number of differences. A first sense of difference in the analogy is that a subject is revealed by its predicates but potency does not directly reveal act. Consider the slight differences between Felis lybica and Felis catus: careful examination is required of the predicates in order to conclude that they are different species, not the other way around. Nevertheless, any given cat is known to be a cat intuitively, without any recourse to laborious argumentation. At best, this is not the case with act; to posit a claim for its real existence requires rational discourse - in other words, it is not directly knowable. A second sense of difference in the analogy is that whilst accidents depend upon their existence by the substance, the dependency is derived rather than immediate, or primary, unlike the existence of potency upon act. Upon the death of a four kilogram cat, there are still four kilograms of something, even if it is no longer a cat, so whilst four kilograms do not exist in its own right, it does have a strength of existence that may persist despite the loss of its original substantial bearer. Potency, on the other hand, just is the fundamental principle of change. Being the principle of all possible change, there cannot be, by definition, a persistence of potency in any sense in the absence of act. Finally, a third sense is that change in predicates may destroy a subject, the principle of stability, such as the death of a cat due to a sudden loss in a critical amount of quantity, whereas change in potencies would have no consequence to act, indeed annihilation of all changeable things would be inconsequential to act by definition. Stating this in another way, potency depends on act but not vice versa. The predicates of subjects depend on its substance, but despite the priority of this dependence there is real cooperation or interdependence. Therefore, it would appear to be reasonable to associate potency to act and accidents to substance by the same relation of `present in', but its interpretation needs to be analogous in order to preserve the real differences between the metaphysical principles. This lack of univocal parsimony in some way reflects the development from fundamental principles true of all changeable beings, namely being compositions of potency and act, to what is true of things as complex as living beings. It is not surprising, then, that univocal parsimony is not preserved.


\section{Returning to potency and act}

The overall argument has examined two stages. The first stage examined whether or not change is a real phenomenon and concluded that it is, but change is not continuous in every way for all things. For change to occur, there needs to be some principle of change. Potency and act were the two principles introduced to account for the reality of change. The notion of bearer of change explained the very existence of things, here and now. This notion was developed by showing how the mind must also be a bearer of change since it has the capacity to develop a rational argument. However, such a limited sense of bearer of change, without further development, would serve little use for narrowing down on responses to specialised questions in philosophy, like the mind-body problem. That change is real, and it can be accounted for by principles of potency and act do not, at least in any obvious way, rule out any of the theories on the mind-body problem considered earlier.

In this chapter, an attempt has been made to develop the notion of bearer of change in order to discern whether or not change, outside the activity of the human mind, is effectively atomic. If it is, then there is no reason to posit hylemorphic composition as a response to the mind-body problem as it would claim too much. Substance dualism, on the other hand, may be a better fit since the mind is what is irreducibly different, a bearer of change beyond simply existing here and now, unlike the rest which is matter. Property dualism may also be a reasonable candidate, since to associate a non-trivial bearer of change with a property rather than an irreducible principle has not yet been ruled out. Physicalism and panspychism could less readily be admitted. These two positions do not appear to require a notion of bearer of change, but as has been seen, it is needed to at least account for the motion of the human intellect.

To progress the case of hylemorphic composition as an account of change in material things, the scope of bearer of change has been widened to not only include the human intellect but also to include living things. This development has sought to deny that change is not atomic for living things but rather is governed by a bearer of change, namely, substance. Furthermore, an analogy to change taking place in a bearer of change has been presented, whereby accidents are `present in' substances, with the former admitting variations without necessarily corrupting the subject. This argument, however, does not suffice to claim hylemorphic composition as a preferential theory, as this theory claims that bearer of change is found in all material things in a non-atomic way, not just living things.

The final chapter will advance the need for positing hylemorphic composition for all material things by arguing that a principle of individuation is required for particularising kinds of things. With the hylemorphic view then presented, various criticisms of the position will be examined and responded to.
