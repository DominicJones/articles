\chapter{Introduction}
\label{ch:chapter1}


\section{The mind-body problem}

The breadth of the natural world comes to a head in philosophy when the question of where man fits into the metaphysical scheme is raised. Is the mind the brain or is it something of a radically different nature? For Aristotle, the mind is able to form concepts, which are necessarily abstract, and so have no dependence on matter, implying a radical difference. (\acrshort{aristotle-de-anima} III.4) Furthermore, this position is not merely historic, but is reaffirmed in contemporary work (e.g. \cite[][250--55]{Oderberg2007-ODERE-2}).
%% \begin{quoting}
%% Nothing abstract exists without abstraction. And abstraction is an intellectual process by which we recognise what is literally shared by a multiplicity of particular things. \parencite[][]{Oderberg2007-ODERE-2}
%% \end{quoting}
If there is such a radical difference, it is difficult to see how mind and brain interact without violating laws of physics, especially those concerning the conservation of energy and momentum.
There is, then, a tension between what is known about the world through the empirical sciences and what is known through one's first person experience of the world.
This problem is what philosophers call the `mind-body problem', and it is roughly the problem of explaining the relationship between mental and physical phenomena. It is difficult to give a more precise statement of the problem that is not biased towards one or another theory of mind.
\begin{quoting}
For instance, to characterise it as the problem of explaining how immaterial mental substances can interact with the body seems to presuppose the truth of dualism; while to characterise it as the problem of explaining how mental processes are produced by physical processes in the brain seems to assume the truth of materialism.
\parencite[][250]{feser2006philosophy}
\end{quoting}

The problem does not hinge on whether or not there is some aspect of mental activity which correlates with neural activity. There may well be a neural correlate for every aspect of the mind; that is a topic of research for neuroscience. (cf. \cite[][]{Kim1966-KIMOTP}) The philosophical question, instead, is whether or not mental processes can plausibly be said to be either identical with neural processes or metaphysically supervenient upon them.\footnote{One thing supervenes on another when there could not be a difference in the first without there being a corresponding difference in the second. \parencite[][]{Haugeland1982-HAUWS}}
However close the causal correlations between the mind and neural activity, such correlations, by themselves, cannot establish identity or supervenience.

Responses to the problem can be divided into those that are characteristically monist and those that are characteristically dualist. Positions of the former kind imply that there is only one irreducible principle and any further phenomena derive from this, whereas positions of the latter kind imply that there are two interacting but irreducible principles.

Arguably the most common monist position is physicalism. This is usually understood as the view that there is nothing over and above physical things {\textemdash} there exists nothing more than the things described by physics.
\parencite[][]{Papineau2001-PAPTRO}
%% The main argument for physicalism places the emphasis on laws of physics ruling out the possibility of interaction with another other principle.
The main argument for physicalism is that the laws of physics rule out the possibility of physical things interacting with nonphysical things.
Even if other kinds of principles were to exist, such as the \emph{res cogitans} of Descartes, there would be no way to account for the causal interaction between the two without forfeiting the universality of physical laws. A consequence, however, of upholding the universality of physical laws is a severe constriction on the means by which we may to account for the distinctive characteristics of the mind, for which physics offers little explanatory value.

In seeking to correct the lack of appeal of physicalism with respect to accounting for characteristically mind-related capacities, panspychism shifts the emphasis on what matter really is. Physicalism tends to say that matter is that which physics examines. Anything that falls under the methodology of physics is the raw material, or the principle, of physicalism. However, panspychism argues that the mind-body problem arises out of a misunderstanding of matter, and that there is no problem once we fix our understanding of matter, hitherto construed in a too narrow way, as being purely quantitative. Rather, for the panspychist, when matter is viewed as both that which makes up the brain and nervous system and that which makes up sticks and stones, one can argue that consciousness is, to some degree, a real possibility of matter as such, a \emph{quality} of matter, something in addition to its quantity.
(cf. \cite[][]{goff2019galileo}; \cite[][]{Chalmers2013-CHAPAP-17}; \cite[][]{Strawson2006-STRRM-2})

For these first two monist views matter is the focus, and the mind is something to be accounted for through it.
A third view is property dualism which, despite its name, also shall be categorised here as a monist position.
This position appears to sit somewhere between physicalism and panspychism. It, like panspychism, rejects physicalism for its failure to account for all of the evidence related to the human condition.
That there is `nothing over and above' physical state appears to have no way of accounting for the aspects of a conscious experience accessible only from the first person point of view.
(cf. \cite[][]{Nagel1974-NAGWII}; \cite[][]{Levine1983-LEVMAQ}; \cite[][]{Jackson1982-JACEQ}; \cite[][]{Chalmers1996-CHATCM})
However, it rejects panspychism for not recognising there is something distinctive of animate organisms, rather than universal to all matter, about the properties of the mind.
\begin{quoting}
Mind and matter are equally fundamental aspects of reality, neither reducible to the other, [...] but there is one one kind of substance, namely physical substance. {[Property dualism]} holds that physical substance nevertheless has two fundamental kinds of property, namely, physical properties and mental properties.
\parencite[][244]{feser2006philosophy}
\end{quoting}

Two characteristically dualist positions are hylemorphic composition and substance dualism. The first of these two positions is not one specifically constructed to respond to the mind-body problem but rather is first of all an account of all materially existing things.
According to this view, all material things are individual things of some kind, and matter is the principle of individuation while form is the principle of being something particular.
(cf. \cite[][]{Oderberg2005-ODEHD}; \cite[][]{feser2014scholastic})
For the hylemorphist, the mind is not a principle but is \emph{present in} the form of the human person, and the body is matter marked by quantity, and coordinated and unified by the form. These two irreducible principles must, for all material things, be united for either to exist. Hylemorphic composition is the position which this work will focus on, attempting to provide an adequate argument for its credibility.

Moving towards greater independence of principles is substance dualism. This theory says that there are two fundamental kinds of substance, namely mental and physical substance, where a distinctive characteristic of the position, as commonly understood, is that it maintains that minds can exist in the absence of bodies and vice versa.
Defining significant differences between some interpretations of substance dualism and hylemorphic composition, however, can be far from straightforward, since even existential interdependence of principles has been affirmed in the context of substance dualism.
\parencite[][]{Lowe2006-LOWNSD}


\section{Composition in change}

The question of the nature of man is urgent, but the manner in which it is generally approached in contemporary philosophy, I suggest, is `too close to the eye of the storm' when not already equipped with sure ontological foundations. An example of the confusion caused due to a lack metaphysics grounded independently of the mind-body problem can be seen in \textcite[][]{Searle2002-SEAWIA}. He seeks to clarify his position as someone who does not accept to property dualism, but despite this, \textcite[][]{feser-2004-searle} argues convincingly that Searle has only further emphasised his position as a property dualist.
In order to establish the requisite ontological foundations, this work proposes to reconsider an old starting point that seems to have fallen out of fashion, the phenomenon of change. It is a starting point that should put sufficient distance between the enquirer and the phenomenon such that a reasonably secure metaphysics can be laid out, one which could provide the groundwork for more specialised questions, such as the mind-body problem.
(cf. \cite[][]{sep-change}; \cite[][]{sep-temporal-parts}; \cite[][17-68]{feser2017five})

Turning a statement from Aquinas (\acrshort{aquinas-summa-th} I.9.1) into a question captures the essence of this work: is it the case that `in everything which is moved, there is some kind of composition to be found'?
Though the question may seem a long way off from something as particular as the mind-body problem, there is an obvious connection.
For, if change necessarily involves composition, it will follow that there exists more than one fundamental principle. This multiplicity of principles might then provide the general basis for mind and body.
Aquinas treats movement as a robust experiential premise and concludes that composition is found in change, specifically hylemorphic composition in the case of material beings, including man.
(\acrshort{aquinas-summa-th} I.76.1)
In order to establish the credibility of Aquinas's claim, it will be necessary to engage in three broad areas of enquiry:
is change ontologically real; does change presuppose a bearer of change; is a material bearer of change hylemorphic?

\subsection{Is change ontologically real?}

That some things change seems like a common-sense truth about reality that is in no need of questioning. Yet, on the outcome of its analysis rests the core of Aristotle's metaphysics, namely that composition of potency and act encompasses all of reality, and furthermore, the composition of matter and form, as a specific case of potency and act, encompasses all material reality.
Aristotle's hylemorphic composition has very little acceptance in contemporary academic philosophy. This may be because of the weakness of the premise that change really exists or the premise that any change is caused by something already actual, or else because of some error in his argument from change to composition. But because change seldom appears in discussion that touches on critiques of hylemorphic composition, it is not obvious where exactly the problem with Aristotle's metaphysics lies.

Historically, however, change was a central locus of philosophical enquiry, most notably in the work of Heraclitus and Parmenides, the former arguing that everything is in the process of change, and the latter arguing that change is impossible. Either position would pose a serious challenge to Aristotle's argument from change to composition. Examining the arguments, especially under the interpretation of Cohen (\citeyear[][]{cohen-heracli}; \citeyear[][]{cohen-parm1}), it can be seen that refutations of either position require significant work. Even if sound arguments can be presented for the reality of change, ancillary work about its nature raises further important avenues of investigation, such as: is the changer necessarily something changing, too;
\parencite[][17-68]{feser2017five}
is there such thing as instantaneous change;
\parencite[][]{Oderberg2006-ODEICW-2}
are there infinite actual mid-points of a change, as exemplified by Zeno's paradox; is it the case that everything that changes has an agent of change;
\parencite[][]{Oderberg2010-ODEWIC}
and are the compounds of change ontologically separable compounds in some cases, or do they only involve `real distinctions' (to use the traditional term)? The question of change leads to many related questions that provide the means for laying out baseline metaphysical principles, like causation and substance, principles which seem indispensable for more specialised philosophical enquiries.

\subsection{Does change presuppose a bearer of change?}

When we talk about something changing, like a tree growing, we imply that change takes place in something; something stands under the change and persists through the change. In the case of the tree, it is the same tree that was once a sapling but has now reached maturity.
Intuitively, we must posit an essence of this particular tree, its `oakness', to be the bearer of its growth. However, when the tree dies, this is also a change, going from being some particular tree to no longer being any tree.
What then is the bearer of this change? Since it cannot be `oakness', bearers of some changes, like growth, cannot also be bearers of all changes, like growth and death. Bearers of change, then, appear to be principles of relative stability, but not principles of permanence.
However, if `oakness' is not something permanent, is it therefore necessary for something to be its bearer of change; that is the bearer of the changes it undergoes, but of which it cannot be the bearer?
Recursively applying this question would appear to demand a termination, namely something that does not undergo change at all.

Considering the question of change not only provides the opportunity for establishing metaphysical primitives, forming something like a toolbox for working on the mind-body problem among other things, but, far more importantly, it provides an opportunity for positing \emph{the} metaphysical primitive. For, change is the manner in which Aristotle approaches the existence of God, the first principle of change that itself does not undergo change.
(\acrshort{aristotle-meta} XII, 1072a)
Whilst Aquinas compiles four other ways in which to approach the existence of God from purely rational discourse, the way favoured by Aristotle appeals to common-sense experience, namely, that some things change, and so is perhaps the most accessible and compelling of the five ways.
(\acrshort{aquinas-summa-th} I.2-26)

Two questions intervene between the reality of change, and the need for a fundamental, traditionally divine, principle. First, for change to be considered real, does a bearer of change need positing? And second, if so, does the bearer of change in turn require a bearer, ultimately terminating in an unchangeable bearer of change, or is it adequate for the bearer merely to be a principle that is more stable than the change it bears?

The connection of change with traditional arguments for God means that this topic is by no means of purely theoretical interest. Indeed, in a critical book review of `Five Proofs of the Existence of God' \parencite[][]{feser2017five}, \textcite[][]{Blackburn2018TLS} argues that the ascent up to God, arriving at the `timeless, sunless realms' of `subsistent existence itself' is an exercise in a `dazzling and deceptive illusion' as its supposed gains in fact turn out to be empty-handed metaphysical catchphrases having no `more content than a vacant ``something-we-know-not-what'''.

Blackburn recognises that if such arguments were to establish philosophical knowledge of God then they might also ground the dictates of a specific morality, a morality that is enforceable, not on religious grounds, which can readily be dismissed into the so-called private sphere, but on rational, and thus universal, grounds. Even if the argument is successful, Blackburn would resist this on the grounds that we have `no use for any conception of God', because `we can never, according to the rules of just reasoning return back from the cause [God] with any new inference'. It appears that for Blackburn, whatever merit the proofs offer in getting us to that which we commonly refer to as God, attaining this conclusion is vacuous due to its lack of intelligibility.

Blackburn expresses scepticism about philosophers' claims to understand the conclusion that God exists (i.e. that there is an `uncaused cause', etc.) and to rigorously develop the implications, especially dictates of morality. Engaging in rational inquiry into something so elevated as to be utterly opaque, and then forming conclusions which lend themselves to being enshrined in dogmatic imperatives, ought not have a place in philosophical discourse, according to Blackburn.

In response to the review, \textcite[][]{Feser2018TLS} argues that if natural science is to be intelligible it needs to be sufficiently backed up with a robust metaphysics. This, Feser claims, will involve recognising the conservation of a substance through its accidental change, assuming that the totality of causes are found in their effects, that a cause cannot produce just any effect but a limited range of possible effects, etc. But in backing up natural science with such a metaphysics, Feser argues, it can also be shown, albeit with difficulty and only narrowly at that, that rational accounts of the existence of God can be derived, and furthermore the essential properties of God expounded. Feser argues that `all concrete reality is intelligible', and that to deny this either by denying that only some or else no concrete reality is intelligible would either be to select arbitrarily those things that fall into the capacity of reason or else to utterly deny any kind of scientific enquiry. Feser advocates pushing reason to the very limits of its natural powers, and defends this as a robust way of reaching knowledge of God, drawing upon the methods of the \emph{via negativa} and the analogical use of language. Furthermore, he argues that to deny the robustness of the arguments used would also be to deny the foundations of scientific knowledge.

I shall argue, following Feser's interpretation of Aristotle, that the question of change leads to the conclusion that an unchanging principle that motivates change must be posited. Doing so is not a side topic en route to establishing the ontological nature of material things, but a question that cannot be avoided, and the manner in which it is answered considerably narrows down possible responses to the mind-body problem. Indeed, I shall argue that only one answer is left standing.

\subsection{Is a material bearer of change hylemorphic?}

The four causes, as presented by Aristotle, offer a way of giving an account of natural things, and are often presented through the exemplar of a living thing.
(cf. \acrshort{aristotle-phys} II.3; \acrshort{aristotle-meta} V.2)
The four causes, or four explanations as it is sometimes translated, are the material and formal causes (hence \emph{`hyle'-`morphe'}, meaning `matter'-`form') which answer the questions `what is it made of?', and `what kind of a thing is it?', and the efficient and final causes which answer `how did the change come about?', and `what motivated the change?' They offer a way of giving an account of substances, providing distinctions to communicate common-sense characteristics framed at a minimal level of philosophical precision. For example, considering a cow in its typical habitat, four perspectives on it may be adopted: that it is flesh and bone, specifically a domesticated breed of ox, bred at this particular farm, in order to produce meat for the farm. The account is holistic, offering a means of identifying the substance, its origin and its purpose. However, it is partial, as greater precision could be provided in the account, such as that the cow is of the female sex, its hooves are cloven, etc. Such precision could go on indefinitely, though what is typically required from an account is enough detail to identify and interact with the substance in an intelligent way.

Like change, the four causes appear to be in no need of a defence. They can be seen as common sense spelled out in marginally technical terms.
But despite this, they have generally come to be regarded with scepticism with respect to their ontological reality, even if many philosophers continue to acknowledge their explanatory power.
\parencite[][]{Robinson2018-talk}
The formal and final causes tend to receive the strongest critiques among the four causes. Unlike the material and efficient causes, which seem more immediately part of ordinary experience {\textemdash}what things are made of and how things work{\textemdash} the formal and final causes can, at best, appear to be narrowly credible, and, at worst, appear to be absurd ideas. The formal cause is something that cannot be extracted and examined and yet it is said to inform matter. The idea of a final cause posits that both animate and, to a lesser extent, inanimate things have directedness, giving credence to the thesis that even sticks and stones tend towards certain ends.

The main focus of this last subsidiary question, however, is matter and form. These two causes are argued, by hylemorphists, to be distinguishable but not capable of existing separately, unless the form itself has capacities that are independent of matter, since, according to Aristotle, it is the substance itself that exists, i.e. that which is composed of matter and form. But, if form is not separable, and yet dictates the nature of a substance, where does it reside? If the answer were that it resides in the matter itself, for example as DNA in the case of the cow, the form would be nothing more than a specific detail of the matter, and so form can be dispensed with. Alternatively, if it is simply ascribed to the brute fact of the substance itself, that it is this way but could have been something very different, no explanatory power is found in the distinction. But for the Aristotelian, there is a real distinction that characterises and unifies and yet is out of reach of the experimental probe. Not surprisingly, then, this is a position that is readily suspected of metaphysical casuistry.

In eliminating form, matter becomes that which is real, and scientific study tends to become the leading authority on what the nature of things really is. Distinguishing between natural kinds, especially living things, and artefacts must then concern a difference in degree of complexity rather than irreducibly different kinds.
This third part of the discussion will present a case for hylemorphic composition, and respond to some of the competing arguments.

This third subsidiary question may appear to be misplaced with respect to the first two. First, the experience of change is considered, and then following this step, abstractions from change are made that are common to all things that change and bear change. The first two stages clearly go from the particular towards the universal. However, in this final stage, material things are brought back into focus, a stage that would seem to fit more appropriately between the first two rather than after the second in order to chart a step-wise advance in abstraction. However, despite the seemingly out-of-order sequence, the placement of this third step is arguably in keeping with the manner of knowing:
\begin{quoting}
  In order to understand the concrete singular as such, the intelligence, after having formed the universal concept, returns to the original experience that has led it to the universal, and superimposes the universal on the concrete beings: this operation is called `return to phantasm', and it leads to further `in-formation', i.e. to increasing knowledge, always in close contact with reality.
\parencite[][170]{de1981christian}
\end{quoting}
The discussion of the material bearers of change in this third part may be considered a `return to phantasm', and is advanced as the means for corroborating the abstractions developed in response to the previous subsidiary questions.
