\documentclass[12pt]{article}
\usepackage{fontspec}
\usepackage{sectsty}
\usepackage{geometry}
\usepackage{csquotes}

\title{Holiness and Chastity\\ \vspace{5mm} \large{Latin Mass Society, Iota Unum series, London}}
\author{Dominic Jones}
\date{31 January 2020}

\begin{document}
\maketitle

\section*{What's not in the talk}

This talk is more of a `depth rather than breadth' approach to the topics of holiness and chastity. Consequently, there are important areas missing. So, in order to offer pointers to other talks that would help fill in what's missing, here are some suggestions.

First, a talk by Ed Feser, titled `Cooperation with Sins Against Prudence', available on the Thomistic Institute. He has written a fair amount on the topic and can be found on his blog by searching `Love and sex roundup'. He deals with the perspective of the natural ends of our sexed state, and expounds the sins against prudence with respect to lust. The latter topic is otherwise known as the `eight daughters of lust'.

The second recommendation is the various talks by Fr Philip Wolfe on the website `Veritas Caritas' under the section on Commandments. He offers plenty of sound, practical advice.

I don't recall any particular talk on holiness standing out so as to recommend; partly because I don't recall coming across any, but even if I had I would have probably passed them over as expecting them to  be a bit dull. They just can't compete with talks like `Signs of the apocalypse'. That said, I think Fr Edward van den Berg's talk the other month on spiritual direction was a very pertinent to growth in holiness. I don't think it can be swapped out with something else and the same end, or rather, degree of end, be attained.

\section*{Introduction}

The path of holiness, and therefore chastity too, is not so open-ended so as to admit its attainment 'by any route'. St Paul lists some of those routes that do not reach heaven.
\begin{displayquote}
Know you not that the unjust shall not possess the kingdom of God? Do not err: neither fornicators, nor idolaters, nor adulterers, Nor the effeminate, nor liers with mankind, nor thieves, nor covetous, nor drunkards, nor railers, nor extortioners, shall possess the kingdom of God.
(1 Cor 6:9-10)
\end{displayquote}
What we have are the Ten Commandments and we can readily learn of the wider implications of those laws. We also know from Our Lord that he who wishes to be saved must keep the Commandments. At first glance, things are pretty straightforward. There is a definite lower bound and if we obstinately persist in living beneath it the outcome is clear.

Despite this clarity, like the rich young man of the Gospels, we tend to grope around for something more than negative precepts. Our lives are not lives merely for `not doing'. When we reflect on our own situation will are well aware that there is a `something' that we are not but should be, like `Beacons of holiness' and `exemplars of chastity'. We know, but may readily just leave it at that. We know that the rich young man went away sad even though he was told, implicitly, that he would inherit eternal life. In some way, holiness and chastity are like will-o'-the-wisps; `If I just do `X' then I shall finally be chaste or holy, or even both. But what is this elusive `X'?'

Talks on holiness and chastity may attempt to offer the magic words to fill the unknown, and this talk will be no exception; I shall offer my sprinkling of magic words, too.

But before we get onto magic words, I wish to preface it all with the fact that there is simply no substitute to knowing the good to be done and the evil to be avoided, and knowing it as perfectly as can be attained. Proposing that someone gets familiar with the law as a sure means for kicking into touch his illicit romance may seem a little futile. On its own, the law seems unable to compete. Human experience seems to tell us there is something else in the mix, something that doesn't yield readily to the formulas of knowledge. This something drives us to act, not infrequently throwing caution, and law, to the wind.

Eleonore Stump captures the heart of the tension between what we know to be good and what we want as good for ourselves. On the one hand the good to be done and the evil to be avoided are simply general principles which are essential to our flourishing as human beings. On the other hand there are the desires of our hearts, the goods we desire as goods for ourselves. And the two are not the same; one is not reducible to the other. The latter is not simply a damaged way of viewing the former, one which only makes its appearance in us after the Fall. Nevertheless, some flavour of dualism is not being proposed here. The unity is in me: I am a human being and I have these desires of my heart. Holiness is neither in the intellect nor is it in the will. It is I who am either growing in holiness or forfeiting holiness. Recognising these two determinations, the essential flourishing of our nature and desires of our hearts, should help us get some grip on the work of our growth.

\subsection*{Hierarchy of determinations}

Saying that these two determinations are irreducible is not to say they are not in a hierarchy. What I can conceive of outstrips what be achieved in reality. And if I can conceive of something I can desire it. I can desire the recovery of someone close who on the cusp of death due to an illness, but, nature taking its course, this person will surely die soon. Generally, these kinds of desires, however, bear no strong relation to the work of holiness. A desire that isn't possible is not going to have much impact.

Much more the difficulty are those desires that are physically possible but morally prohibited. An example of this is in some way captured by a remark made by Elizabeth Anscombe on modern contraceptives. She wrote that the state of affairs in the realm of sexual morality has simply changed forever. Something that was not possible, and so never thought about, is now possible. And now it is possible it is thought about and desired. The move from impossible to possible radically changes the landscape of desire. Once upon a time not adhering to `Thou shall not covet thy neighbour's wife!' came with enough weighty consequences in the natural order that pondering the supernatural consequences just wasn't all that necessary. But with the dawn of `possibility', and a very weak sense of essential flourishing, desires of the heart capitulate to self-justification. `Ah yes, but I love her! We were made for each other! This can be worked out...'.

The state of affairs now is, more or less, that most things are possible, and, secondly, very little is known about the good to be done in order to flourish. This is a good a cocktail as needed for things to go rather badly rather quickly. At the social level, nobody is the wiser if the woman I may live with is my wife or just another woman. At the personal level, nobody knows and whether or not I frequent the virtual brothel I carry around with me all day everyday.

Two fronts, then, need re-enforcing: growing in sure understanding of the good to be done and the evil to be avoided, and purifying the desires of our heart. Some of this ground can be covered by study, some of it by the sacraments, but there is another chunk that is not so receptive to a `practice', namely purifying the desires of the heart.

\subsection*{Purifying desire}

Somehow there is an interior work of alignment whereby our flourishing is not frustrated by our desiring. Perhaps the alignment of flourishing and desiring could be seen hierarchically when imagined like this. Suppose the inside surface of a cone marks out the domain of natural law. Within this cone an act is permissible. Wrapping that cone is a wider angled cone. This cone marks out the domain of physical law. Within this cone an act is possible. Living a chaste and holy life might be imagined to somehow realigning the arrow of our desire, and so our action, from being sandwiched between these two cones to being inside the inner cone. This task might represent what is needed of the `coveting man'. However, life is not that simple and there are other, narrower, cones still. A cone within the natural law might be the bond of marriage. An arrow of desire that may have once been happily lying within the innermost cone now finds itself sandwiched. Once upon a time it was perfectly fine to take a woman out for dinner, but now it isn't. A new inner cone demands greater reserve to those who are not one's wife.

Even if fairly procedural answers can be given to better knowing the good to be done, no simple answer is offered to the much more difficult work of purifying our desires. Eleonore Stump did offer an answer, lifted straight from the Psalms:
\begin{displayquote}
Take delight in the LORD, and he will give you the desires of your heart.
(Ps 37:4)
\end{displayquote}
But how do I take delight in the Lord when it is His law that puts me on the wrong side of holiness? Or how do I take delight in the Lord when the prudential path runs counter to my desire? Regardless of the kind of conflict that the desire of the heart presents, nobody is spared the difficulty of attending to its resolution.

\subsection*{Struggling with God}

It seems to be a struggle that if met facilitates the passing from a formulaic belief to personal knowledge of God, somehow a growth in holiness that cannot simply be realised by the work of learning alone. The mysterious account of Jacob wrestling until dawn offers a metaphor for the kind of contact God wishes us to have with Him.
\begin{displayquote}
Jacob was left alone. And a man wrestled with him until the breaking of the day. When the man saw that he did not prevail against Jacob, he touched his hip socket, and Jacob's hip was put out of joint as he wrestled with him. Then he said, `Let me go, for the day has broken.' But Jacob said, `I will not let you go unless you bless me.' And he said to him, `What is your name?' And he said, `Jacob.' Then he said, `Your name shall no longer be called Jacob, but Israel, for you have striven with God and with men, and have prevailed.'
(Gn 32:22 ff)
\end{displayquote}
In a not so dissimilar way, Our Lord, too, wrestled with God, wrestling one will with another will. Here it is not a case of law against desire, but natural, personal desire against a greater good but one laden with personal loss.
\begin{displayquote}
And going a little further, he fell upon his face, praying, and saying: My Father, if it be possible, let this chalice pass from me. Nevertheless not as I will, but as thou wilt.
(Mt 26:39)
\end{displayquote}

It is hard to see how phrases such as `whatever God wills' really reflects a work of holiness. Such phrases seem to indicate not having a will in the first place, or pretending that desires of the heart are non-existent, as though complete passivity is something proper to a bearer of free will. Jesus did not blandly say in the garden of Gethsemane `whatever God wills', but he wrestled with God. And I too must wrestle with God, coming into contact with Him in a way that is unavoidably confrontational, and so unavoidably personal.

\section*{Holiness}

Of the three immaterial faculties of the soul, the memory typically is regarded as bottom of the pile. Intellect is king, the will, that's king too!, but the memory, well, let's not think about that too much. But memory will be the motif for approaching the topic of holiness. This faculty which, unlike the intellect and will, straddles the the material--immaterial divide,  cannot be neglected in our response to the call to supernatural life.

\subsection*{Remembering}

The Church is not silent about our much needed use of this seemingly humble power. Of the few times she addresses me personally, on one occasion she does so with these words: `remember that you are dust and to dust you shall return'. Not the most cheery of lines, but I would do very well indeed to remember this. Two other references to memory also come to mind that we find in Sacred Scripture. After Jesus seemingly reprimands Mary and Joseph for agonising over his disappearance in the Temple, Our Lady's response is recorded as `she pondered these things in her heart'. There was no answer, no formulaic explanation of the dialogue. The ineffable mystery of God pushed onto the scene forcefully, albeit briefly, and there are no words to resolve what Mary and Joseph were confronted with. The other is the good thief. In his last words he asks to be remembered. Not, `remember all the good works I have done...', but `remember \emph{me} when you come into your Kingdom'. There was no point talking about justice and what was owed whom.

In addition to these particular examples of remembering, I think there are two more which the Church wishes us to consider. The first is the corpus of the Church’s memory, exemplified by Sacred Scripture as a whole, and also by the memory she fosters of her saints.

Sacred Scripture is the complete salvation history, and to have a grip on its broad strokes would serve to anchor us as part of that history. I am surely here to not only seek my own salvation but also, and inseparably, to bind the propagation of Salvation from one generation to the next. This is not merely a goodwill gesture. My historical significance depends on it; in some way my worth depends on it. The \emph{how} we work out for ourselves, but as for the standard, the Church points this out to us.

In England we are endowed with many saints. Their memory is preserved not only so as to give glory to God but also to encourage us. Consider St. Margret Clitherow, to take one example. She was condemned alone and died alone. For her, the consolation of friendship failed at two critical moments: when she refused to be tried by the law of the land, and when the weights were loaded on her to crush her. This latter occasion is poignant. She asked a close friend to watch with her whilst she was to be crushed so that she would not die alone merely in the company of those who wish her dead. Her friend refused, not wished to bear the sight of the execution. And so Margret died alone. And the Church wishes us to remember this; the suffering which she knows that The Father does not spare His Church.

The second is perhaps tenuous, but real all the same. It is remembering our own lives. St Augustine in his Confessions does not gloss over this but retraces his life, and, rather amazingly, seems to be able to take us back to sometime around his birth. The remembering of a factual account is not the sense of remembering of one’s life that I wish to stress. Rather, it is more like going over the faint pencil lines of salvation so that they can more readily be seen. It is not uncommon to come across Catholics who have grown up in a practising family but for whatever reason no longer practice themselves, even if for quite some time as adults they were diligent. What could be appealed to so as to encourage them back to the Church? I think part of the response is remembering those pencil lines traced out by the providence of God. Somehow, I think such a practice would encourage a person in that situation to make a judgement that either causes them to reject the inheritance that her family offered to her as something of a fraud, or to remember the good she once had and knew, a nostalgia that may motivate a change of heart.

\subsection*{Exercising memory}

Pinckaers, in `Sources of Christian ethics' distinguishes freedom into two varieties: freedom of indifference and freedom for excellence. Freedom of indifference consists wholly in a choice between contraries and is separated from all the actions preceding it or following it. Freedom for excellence, on the other hand, consists in habitually ordering acts with the perspective of the final end of the human person, happiness. Freedom of indifference might be likened to `making it up as you go along'. In this approach to life there is nothing to be seen that lies beneath the plain and obvious surface of things. It is a life absent of the need for `pondering in one's heart' as there is simply nothing to ponder.

But this is not the way the Church sees things. At a week to week level, there is a demand by the Church that each one `ponder in his heart' if it is fitting to present one's self for holy communion. Similarly for confession, the Church exacts a `pondering of one's heart' as to the sins he has committed. Therefore, at some minimal level the Church necessitates a `freedom for excellence' approach to governing one's life, whose first requirement is to engage one's memory about one's self.

Whilst there is a straightforward sense to remembering, I think there is also a more subtle sense by which the advancements in holiness may be made. This is the kind more like `she pondered these things in her heart' rather than `remember that you are dust'. The `remember you are dust' kind seem to point more towards the first example; recalling the week gone by and examining one's conscience on the acts carried out.

But before moving on to the second kind of remembering, there is something to point out in the first. The Church expects us to remember. That is, she expects us to be able to recall sufficiently well our thoughts, words, actions and omissions between one Communion and the next, between one confession and the next. There might be an obvious objection to this. `I could never recall \emph{everything} I have said or thought in the week. I might, at a stretch, recall what I have done. And as for omissions, just forget it!' But why would I not be able to recall every thought I have had, at least as towards its moral quality, over the last week? Might it be because I haven't actually thought but rather inoculated myself against such things with the Youtube tranquilliser? Or I have thought, but it has been so light in touch as to barely register in my memory. Why is it that I cannot remember everything I have said, nor the choice of words, the manner of phrasing? Is it because I regard such things as irrelevant to the persons I address, that bothering to put thought into it is only laughable? Or perhaps I have said lots of things, but none directed towards any kind of `freedom for excellence', as if there is some ultimate good to which I am at service. And yet, no one will be admitted into God's Kingdom until he is purified of every last imperfection. God remembers. And we would do well to do the same.

But there is this second kind of remembering, one that is not so much bound to my actions or circumstances, but is more related to wanting to see He whom we do not see. I say that I believe in God, and that he sustains in being all things, that he directs all things towards a greater good, that His will is that all men be saved and come to the knowledge of the truth. Furthermore, that Jesus Christ is the head of the human race, its Redeemer, and my Redeemer. Somehow, after boiling things down, the bottom line is that God seeks my friendship and, somehow, I may seek His. And this seems to me to be the essence of the second kind of remembering. Friendships build up a reserve of history, that is combed from time to time. We reminisce. And we do a lot more of it when the wind of sorrow picks up. When some behaviour is inexplicable, when someone `goes their separate ways', we `ponder in our hearts'. Why? because it matters! It matters to me to try to see the heart of things about that which I care about the most. What did Our Lady think when Our Lord seemed to rebuke her? For some reason, the child she had nurtured those years, suddenly addresses her as in an unfamiliar tone and manner about something she knows nothing of. Perhaps she wondered what she had missed all these years. Was there something glaringly obvious that passed her by? What was it? `And so she pondered these things in her heart'.

\section*{Chastity}

\begin{displayquote}
It moved its head toward hers and told her that, if she would eat of the fruit of that tree, she would become free, and understand how the multiplication of the human race was to be effected. Adam and Eve had already received the command to increase and multiply, but I understood that they did not know as yet how God willed it to be brought about. I saw, too, that had they known it and yet sinned after that knowledge, Redemption would not have been possible.
(`The life of Our Lord', A. C. Emmerich, I, p. 13)
\end{displayquote}

A conversation could be imagined between Adam and Eve after God told them that they were to increase and multiply... `You know what, Adam, I haven't got a clue how this is going to work! I mean to say! Where do we even start!' `Well, all I know was that I was made out of dust, so I suppose we're at least in no short supply of that.' `Yeah, well that's fine if we only what blokes around. What about more like me? What was I made out of?' After a long pause, thinking about the future, feeling the dint in his rib cage and how many ribs he has left, `You know what, I really don't know...'.

There was something unbelievable about the command. They were told to increase and multiply, and somehow, if they were given the command it was because they had the power to realise its fulfilment. It seems impossible to place ourselves in the shoes of mature innocent knowledge. To have the immense power of procreation but not have been told how to exercise it makes for a credible account of why the tree of Knowledge was sought. Eve, more than Adam, wanted to know. The maternal instinct for children may account for it in some way. But she wanted to know, and to know now. This could not wait. When would God reveal the `how'? How long would the suspense continue? At any rate, coming to know via the tree of Knowledge surely would be no different from coming to know via God's word? But it was different, different to the point that had God revealed how they were to multiply then the nature of the very sin would have changed to become irredeemable.

Whilst Emmerich's account is not in any way authoritative, it offers an interpretation of the Fall that places the conjugal relationship between the sexes at centre stage. It's always been complicated, this business of having the race sexed. Even prior to the Fall! Wading into the complications is not, however, this choice of emphasis to address the topic of chastity, but rather consider what Adam and Eve did not do, they did not wait on the Lord.

\subsection*{Waiting}

Salvation history is laden with waiting. The Israelites waited for the Messiah. Simeon waited for the promise to be fulfilled that he would see the Lord. Martha and Mary waited for Jesus when Lazarus was already dead. Our Lady waited for the Resurrection. The Apostles waited for the Holy Spirit. And I wait, too, for things only God can bring about.

Waiting, on the face of it, doesn't seem to have much to do with chastity. I think it does, albeit partially, and certainly not at a superficial level. The act of waiting might be compared to pointing a telescope at an unremarkable patch of the sky and leaving it there to record away. Waiting is like the gatekeeper to seeing. I must wait, and perhaps a long time, before my dull vision finally detects the finger of God. It is not for nothing that St Thomas wrote of Him `Adoro te devote, latens Deitas'. If waiting is to seeing, perhaps immediacy is to coveting. `If only I waited...', `If only I was willing to temper the thrill of the moment...'.

One of the main attractions of films is that everything happens all in the space of a couple of hours. `The world is fine. The world undergoes some disaster. The world is saved', and all in space of two hours. Amazing! Or, entertaining, at any rate. It's entertaining partly because it is something removed from the world of real-time, where things change at the rate of 24 hours a day. There is a thrill and a kick out of the whole thing, lifting me from the tedium of day to day life. `Entertain me!' could well be the default disposition in anticipation of watching a film. What trails not far behind the call to be entertained is another call. `Arouse me!', and it requires a lot less than two hours. `Arouse me!' because what I know of life is dull, slow and tired. `Arouse me!' because the perseverance involved in getting to know real persons is too long and too open to come to nought. `Arouse me!', because I've got nothing to lose.

But we have got something to lose. That something is not obvious, and I have to patiently look for until the full wait of what I have to lose is seen and overwhelms us.

\subsection*{Dignity}

Dignity is only something I forfeit. I could be humiliated, or treated in an undignified way, but it is only I who choose for my own purposes either that which degrades or that which is in accord with my flourishing. We often use the expression `it’s beneath my dignity' to behave in a certain way.

Dignity reflects our own self worth; that I am not merely a convenience for conversation, a useful employee, a talented but otherwise replaceable manager. I am not merely something but someone, and I relate to other people, or I hope to do so, in the same manner I regard fitting that I be related to.

What I am worth is answered in light of the end I am trying to reach. If the end is noble then so too is my worth. But the end is more than noble, it is divine relationship, seeing God face to face, as a man knows his friend. Here and now we are called to be temples of God, God indwelling in us. This is our dignity.

Because of our state of loss of integrity, our dignity is readily lost light of and forfeited in willing something that appeals to our desire but is contrary to our good. Among our natural appetites, our sexual appetite stands out, an appetite that seeks the highest of pleasures. But appetites don’t care much for dignity. It is I who either care or does not care for what is dignified. Faced with a temptation against holy purity, how may I better respond? To answer this, we need to look more ‘upstream’.

Since action follows desire, and for there to be a temptation there must be a desire or appeal, we need to examine why something appeals to me which is contrary to right reason.

If we simply answer `I don’t know why this appeals to me even though I know it to be wrong' and consider the matter no further, then at best we could live only the virtue of continence. Continence contains the impulse of the passions but at the cost of great tension and effort, like exercising an internal violence by our will over ourselves. This is a taxing, but virtuous, endeavour, draining on our capacities to `operate with both hands free'.
\begin{displayquote}
The goal is for the lion to be tamed not chained. Chained, its strength is useless to its master --indeed, the master’s strength may be used in working the chains--. Instead, the lion is to seek the ends presented by the master.
(`Growing in Holy Purity', Andrew Kong, p. 5)
\end{displayquote}
Continence is an answer, but it is somewhat of a poor man's answer, an answer that lacks a proper correspondence to his call to flourish as a being endowed with God’s indwelling.

The work of holy purity is diminishing the root of the conflict, the desiring of that which is forbidden. As already remarked in the introduction, it is a two-fold work. The first, to get a very firm grip on what is in keeping with right reason and what is not and why not. This is an educational work. `It doesn't matter' is a phrase to be abandoned for its manifest falsehood, because it does matter what I think, what I do in private, what I deliberately omit, because it is the same \emph{me} who lives in society, who has relationships with other people, who will form other people with my ideals and ideas.

Does it matter that I voyeuristically look at an image of someone of the mere pleasure of looking? It does because both you and the other have the common dignity of seeing the face of God. Is this helping achieve that end? What if you were to both meet in heaven, and you were to blurt out, `oh, I recognise you!' `But how? We've never met before!' With eternity to explain yourself, and can only tell the truth, the situation is a bit tight!

What does it mean to meet somebody? It means being vulnerable, possibly ignored, possibly rejected. But also, and ordinarily, possibly brought into a meaningful relationship, one that requires time, effort, bringing with it its sorrows and joys.

\subsection*{Passions}

Passions are a passive response to a perceived object. If we let our attention focus on a sexual object, predictably our desires will be aroused. As passion grows so too does attention. Since our desires in themselves do not respond fully to voluntary commands, the passion cannot simply be stopped by willing it to do so. However, attention is much more under the dominion of the will. To give someone our attention, we really have to \emph{give} it. Our passions seek attention so as to turn our attention to satisfying them. But no passion, whether desire, fear, anger, etc., can remain intense without reinforcement. If we refuse to give our attention to illicit passions, because we have understood why, they will diminish. Cravings may need to be endured, but they eventually fatigue.

Problems with impurity can in some cases be thought of as the reaction the heart takes when not being filled in its deepest need for intimacy and self-giving, becoming unfulfilled and self-centred. But when we set our attention on God and learn to place our hope in him, we set out on the path of transformation. To turn to God leads the heart to desires and affections that are incompatible with impurity. As the desire for God grows, we grow in the desire for the things of God, and our aversion to things undignified grows too. It is then love for God that is the second aspect for growing in holy purity. It is love for God that directly produces holy purity in the soul.

\section*{Summary}

The work of growth in holiness and chastity was considered as a work of aligning two fronts: the good which essential to human flourishing and the good I seek for myself, the `desire of my heart'. The good of human flourishing was left to one side, and instead aspects of realigning the `desires of the heart' were considered.

The work of holiness was considered under the aspect of remembering and the work of chastity was considered under the aspect of waiting. To remember that `we are dust' and that we need to learn to remember to account for our lives before God, with precision. Also, to remember in a way like Our Lady pondered: God is mystery and my life in relation to God is wrapped in mystery; He seeks my friendship, despite everything, even though this truth doesn't stare me in the face. Secondly, to wait. The life of being entertained is soon the life of being aroused. Instead, wait and see that God has bestowed on us a dignity that only life in real-time reveals. Dignity waits for her reward because her reward is the cause of her dignity.

\end{document}