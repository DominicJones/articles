\documentclass[a4paper,10pt]{article}
\usepackage[english]{babel}
\usepackage{geometry}
\usepackage{fullpage}

\title{Michelangelo's `David'}
\author{Dominic Jones}
\date{\today}

\begin{document}
\maketitle

\begin{abstract}
This essay considers Michelangelo's David as subject matter for shedding light on the question of what are essential qualities of art of  aesthetic value, what are its prerequisites, what is its end. The answers to these questions are made in light of "Meeting with Artists in the Sistine Chapel" by Benedict XVI, and spending an hour staring at the statue of David in the Victoria and Albert museum two weeks ago.
\end{abstract}

\section*{The figure of David}
The point of entry to the question of whether or not a work is of aesthetic value begins with the sense perception of object itself. Immediately conveyed is the nakedness of David. This is not the nakedness of many other statues: imprecise in figure and detail, or concealing in a work that is precise. Rather, it is the fully revealing nakedness of a man in the prime of his youth, well defined, proportional. This is the first and most striking quality of the work, and which arrests the attention of the onlooker. What maintains the attention of the onlooker is, however, that this is not in fact important as an end, but only as an entry point. After seeing the statue as a whole, two details emerge in light of his nakedness: that his right hand is disproportionately detailed relative to the rest of his body and also his face is very clearly given lines of expression. Once the whole statue is recognised, the final qualities are seen: he is young, of physical perfection, but most importantly is that he is in a state of attention: his slightly contorted body, his hands holding a weapon, and above all his frowning gaze, these put aside the fact of his nakedness, as it has served its purpose. Now the onlooker can enter into the dialogue of the work: Who is this man, on the brink of a calmly executed but devastating action?

\section*{Cultural backdrop}
He is David. And upon this rests the value of this work. If the onlooker does not know who David is, he cannot \emph{know} this work before him: he cannot enter into its dialogue. The artist presents the turning point of the life of David, capturing it in the moment of a resolved decision and absolute trust. If the onlooker fails to see this, then the expression of his face, the whole expression of his body, looses the meaning it wishes to convey: the art of the work cannot appeal to the highest level of its intention. This necessity of a common cultural heritage is the medium for the artist to engage with his audience. Furthermore, it is not a zeitgeist like knowledge but a knowledge and appreciation that is expected to be found in the audience of the sixteenth century and as much so now in the present day. It is a genuine cultural knowledge of the history of mankind that is being appealed to.

\section*{Relative value of the artist's details}
Side to the overall work, there are details to which there is no obvious answer as to why they are the way the are. David is uncircumcised; this is very surprising given he is ``God's chosen'', king of Judea, a Jew. Secondly, is the disproportionate detail and size of the right hand. These details cannot be considered as oversights or deficiencies of the artist's ability. However, the point is that neither do they draw the viewer away from the work as a whole. There will inevitably be circumstantial, politicial or cultural reasons why an artist may add this or that nuance to his work. The onlooker needs to discern between what is relevant to the art, and that which is the artist presenting himself. The former is of value and the latter should be given no special attention.

\section*{Where is the artist drawing the onlooker?}
Already noted his the nakedness of the figure, its hands, its face and its posture. In addition the expectation of action has been stated. These, though good, are not the terminus of this work. The crux of the work begins at the realisation that David imminently faces his nemesis, but faces him uprightly, serenely and confidently: not a self-centred confidence, but confidence based on the confidence of another. This confidence of the \emph{other} is an allusion to God. Now the tragedy: how is it that this young, great man who has such confidence in God soon goes onto throwing so much away? How did his personal tragedy come to be? In the face of David is the drama of salvation, and it must impinge on the onlooker and remind him that this David is not a man of fiction: that he is my fellow man, a historical man, as am I. This is art.
\end{document}
